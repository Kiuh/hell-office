% !TEX root = hell-office.tex
% =====================================================================================================================
% Базовые настройки документа
% =====================================================================================================================
\documentclass[a4paper,14pt,oneside]{report}  % Класс документа
\usepackage{extsizes}                               % Поддержка 14pt
\RequirePackage{shellesc}                           % Выполнение shell-команд
\ShellEscape{openout_any=a}                         % Разрешение записи в файлы
\usepackage{lastpage}                               % Получение номера последней страницы

% =====================================================================================================================
% Шрифты и кодировки
% =====================================================================================================================
\usepackage[utf8]{inputenc}                                 % Кодировка UTF-8
\usepackage[T2A]{fontenc}                                   % Кодировка кириллицы
\usepackage{fontspec}                                       % Системные шрифты
\usepackage{polyglossia}                                    % Локализация

\setmainlanguage{russian}                                   % Основной язык
\setotherlanguage{english}                                  % Дополнительный язык

\setmainfont{Times New Roman}                               % Основной шрифт
\newfontfamily\cyrillicfonttt[Ligatures=TeX]{Courier New}   % Моноширинный шрифт
\usepackage{textcomp}                                       % Спецсимволы
\newcommand{\No}{\textnumero}                               % Символ номера №

% =====================================================================================================================
% Настройки страницы
% =====================================================================================================================
\usepackage[left=3cm,top=2.0cm,right=1.5cm,bottom=2.7cm]{geometry}  % Поля
\usepackage{lscape}                                                 % Альбомные страницы
\frenchspacing                                                      % Правильные пробелы после точек
\usepackage{hyphenat}                                               % Улучшенные переносы
\def\hyph{-\penalty0\hskip0pt\relax}                                % Умный дефис

% =====================================================================================================================
% Математика и формулы
% =====================================================================================================================
\usepackage{amsmath}            % Основные математические инструменты
\usepackage{amsfonts}           % Математические шрифты
\usepackage{amssymb}            % Математические символы
\usepackage{amsthm}             % Теоремы
\usepackage{calc}               % Вычисления длин
\usepackage{siunitx}            % Единицы измерения
\sisetup{
  output-decimal-marker = {,},  % Десятичный разделитель
  per-mode = symbol,            % Формат "в секунду" (м/с)
  range-phrase = --,            % Диапазон с длинным тире
}
\DeclareSIUnit{\sample}{S}      % Пользовательская единица "S"
\AtBeginDocument{               % Нумерация внутри разделов
  \numberwithin{equation}{section}
  \numberwithin{table}{section}
  \numberwithin{figure}{section}
  \numberwithin{listing}{section}
}

% =====================================================================================================================
% Графика и таблицы
% =====================================================================================================================
\usepackage[final]{graphicx}                                            % Вставка изображений
\DeclareGraphicsExtensions{.png,.jpg}                                   % Разрешенные форматы
\usepackage{caption}                                                    % Подписи
\usepackage{subcaption}                                                 % Подписи для подрисунков
\usepackage{rotating}                                                   % Поворот объектов
\DeclareRobustCommand{\povernut}[1]{\begin{sideways}{#1}\end{sideways}} % Поворот текста
\usepackage{tabularx}                                                   % Гибкие таблицы
\usepackage{multirow}                                                   % Объединение строк
\usepackage{makecell}                                                   % Многострочные ячейки
\usepackage{longtable}                                                  % Многостраничные таблицы
\usepackage{array}                                                      % Дополнительные форматы колонок
\usepackage{gensymb}                                                    % Символы единиц (°C)

% =====================================================================================================================
% Оглавление и разделы
% =====================================================================================================================
\usepackage{tocloft}                                                            % Настройка оглавления
\setlength{\cftbeforetoctitleskip}{-1em}                                        % Отступ перед "Содержанием"
\setlength{\cftaftertoctitleskip}{1em}                                          % Отступ после "Содержания"
\setcounter{secnumdepth}{3}                                                     % Глубина нумерации (до subsubsection)

% Стили разделов
\makeatletter
\renewcommand{\thesection}{\arabic{section}}                                    % Нумерация секций арабскими цифрами
\renewcommand\section{                                                          % 
  \clearpage\@startsection{section}{1}                                          %
  {\fivecharsapprox}                                                            % Левый отступ
  {-1em \@plus -1ex \@minus -.2ex}                                              % Отступ перед заголовком
  {1em \@plus .2ex}                                                             % Отступ после заголовка
  {\raggedright\hyphenpenalty=10000\normalfont\large\bfseries\MakeUppercase}}   % Стиль: верхний регистр
  
\renewcommand\subsection{                                                       %
  \@startsection{subsection}{2}                                                 %
  {\fivecharsapprox}                                                            %
  {-1em \@plus -1ex \@minus -.2ex}                                              %
  {1em \@plus .2ex}                                                             %
  {\raggedright\hyphenpenalty=10000\normalfont\normalsize\bfseries}}            % Стиль подраздела

\renewcommand\subsubsection{                                                    %
  \@startsection{subsubsection}{3}                                              %
  {\fivecharsapprox}                                                            %
  {-1em \@plus -1ex \@minus -.2ex}                                              %
  {1em \@plus .2ex}                                                             %
  {\raggedright\hyphenpenalty=10000\normalfont\normalsize\bfseries}}            % Стиль подподраздела

% Центрированные разделы (для введения/заключения)
\newcommand\sectioncentered{                                                    %
  \clearpage\@startsection{section}{1}                                          %
  {\z@}                                                                         %
  {-1em \@plus -1ex \@minus -.2ex}                                              %
  {1em \@plus .2ex}                                                             %
  {\centering\hyphenpenalty=10000\normalfont\large\bfseries\MakeUppercase}}
\makeatother

% =====================================================================================================================
% Списки и перечисления
% =====================================================================================================================
\usepackage{enumitem}                           % Настройка списков
\setlist{nolistsep}                             % Убираем пробелы между элементами
\makeatletter
\AddEnumerateCounter{\asbuk}{\@asbuk}{щ)}       % Кириллическая нумерация
\makeatother

% Форматы маркеров списков
\renewcommand{\labelenumi}{\asbuk{enumi})}      % Уровень 1: а), б)
\renewcommand{\labelenumii}{\arabic{enumii})}   % Уровень 2: 1), 2)

% Отступы списков
\setlist[itemize,0]{                            % Маркированный список
  itemindent=\parindent + 2.2ex,                % Отступ элемента
  leftmargin=0ex,                               % Левый отступ
  label=--                                      % Маркер "-"
}
\setlist[enumerate,1]{                          % Нумерованный список (уровень 1)
  itemindent=\parindent + 2.7ex,
  leftmargin=0ex
}
\setlist[enumerate,2]{                          % Нумерованный список (уровень 2)
  itemindent=\parindent + \parindent - 2.7ex
}

% =====================================================================================================================
% Библиография
% =====================================================================================================================
\usepackage[square,numbers,sort&compress]{natbib}               % Стиль цитирования
\setlength{\bibsep}{0em}                                        % Убираем пробелы между элементами
\bibliographystyle{configs/belarus-specific-utf8gost780u}               % Стиль ГОСТ

% Переопределение окружения библиографии
\makeatletter
\renewenvironment{thebibliography}[1]{
  \sectioncentered*{CПИСОК ИСПОЛЬЗУЕМЫХ ИСТОЧНИКОВ}             % Центрированный заголовок
  \@mkboth{\MakeUppercase\refname}{\MakeUppercase\refname}      %
  \list{\@biblabel{\@arabic\c@enumiv}}{                         % Формат списка
    \settowidth\labelwidth{\@biblabel{#1}}                      %
    \setlength{\itemindent}{\dimexpr\labelwidth+\labelsep+1em}  % Отступ элемента
    \leftmargin\z@                                              % Общий отступ
    \@openbib@code
    \usecounter{enumiv}                                         %
    \let\p@enumiv\@empty
    \renewcommand\theenumiv{\@arabic\c@enumiv}
  }                                                             %
  \sloppy                                                       % Разрешаем свободную верстку
  \clubpenalty4000                                              % Запрет висячих строк
  \widowpenalty4000                                             %
  \sfcode`\.\@m
}{
  \def\@noitemerr{\@latex@warning{Empty `thebibliography' environment}}%
  \endlist
}
\makeatother

% =====================================================================================================================
% Колонтитулы и нумерация
% =====================================================================================================================
\usepackage{fancyhdr}                               % Настройка колонтитулов
\pagestyle{fancy}                                   % Стиль страницы
\fancyhf{}                                          % Очистка настроек
\fancyfoot[R]{\thepage}                             % Номер страницы справа
\renewcommand{\footrulewidth}{0pt}                  % Убираем линию внизу
\renewcommand{\headrulewidth}{0pt}                  % Убираем линию вверху
\fancypagestyle{plain}{\fancyhf{}\rfoot{\thepage}}  % Стиль для пустых страниц

% =====================================================================================================================
% Настройки гиперссылок и перекрестных ссылок 
% =====================================================================================================================
\usepackage[final,hidelinks]{hyperref}              % 
\addto\extrasrussian{                               %
  \def\equationautorefname{формула}                 %
  \def\figureautorefname{рисунок}                   %
  \def\listingautorefname{листинг}                  %
  \def\tableautorefname{таблица}                    %
}

% =====================================================================================================================
% Локализация стандартных названий
% =====================================================================================================================
\addto\captionsrussian{
  \renewcommand\contentsname{\centerline{\bfseries\large{\MakeUppercase{содержание}}}}
  \renewcommand{\bibsection}{\sectioncentered*{CПИСОК ИСПОЛЬЗУЕМЫХ ИСТОЧНИКОВ}}
  \renewcommand{\listingscaption}{Листинг}
}                                                                               % <-- Закрывающая скобка для \addto

% =====================================================================================================================
% Настройка вида оглавлений
% =====================================================================================================================
\makeatletter
\renewcommand{\l@section}{\@dottedtocline{1}{0.5em}{1.2em}}
\renewcommand{\l@subsection}{\@dottedtocline{2}{1.7em}{2.0em}}
\renewcommand{\l@subsubsection}{\@dottedtocline{3}{3.7em}{2.5em}}   % Исправлен уровень на 3
\makeatother

% =====================================================================================================================
% Специальные счетчики
% =====================================================================================================================
\usepackage{totcount}                                               % Подсчет общего числа элементов
\regtotcounter{section}                                             % Всего разделов

% Пользовательские счетчики
\newcounter{totfigures}                                             % Всего рисунков
\newcounter{tottables}                                              % Всего таблиц
\newcounter{totreferences}                                          % Всего источников
\newcounter{totequation}                                            % Всего формул

% Команды для вывода итогов
\providecommand\totfig{} 
\providecommand\tottab{}
\providecommand\totref{}
\providecommand\toteq{}

% Сохранение итоговых значений
\makeatletter
\AtEndDocument{                                                     %
  \addtocounter{totfigures}{\value{figure}}                         %
  \addtocounter{tottables}{\value{table}}                           %
  \addtocounter{totequation}{\value{equation}}                      %
  \immediate\write\@mainaux{                                        %
    \string\gdef\string\totfig{\number\value{totfigures}}           %
    \string\gdef\string\tottab{\number\value{tottables}}            %
    \string\gdef\string\totref{\number\value{totreferences}}        %
    \string\gdef\string\toteq{\number\value{totequation}}           %
  }                                                                 %
  \addcontentsline{toc}{section}{CПИСОК ИСПОЛЬЗУЕМЫХ ИСТОЧНИКОВ}    %
}
\makeatother

% Сброс счетчиков при новом разделе
\pretocmd{\section}{\addtocounter{totfigures}{\value{figure}}\setcounter{figure}{0}}{}{}
\pretocmd{\section}{\addtocounter{tottables}{\value{table}}\setcounter{table}{0}}{}{}
\pretocmd{\section}{\addtocounter{totequation}{\value{equation}}\setcounter{equation}{0}}{}{}

% Подсчет источников
\pretocmd{\bibitem}{\addtocounter{totreferences}{1}}{}{}

% =====================================================================================================================
% Пользовательские команды и окружения
% =====================================================================================================================
% Пояснения к формулам
\usepackage{fp}                                                         % Вычисления с плавающей точкой
\newlength{\lengthWordWhere}                                            % Переменная для ширины слова "где"
\settowidth{\lengthWordWhere}{где}                                      % Замер ширины
\newenvironment{explanation}{
  \begin{itemize}[leftmargin=0cm, itemindent=\lengthWordWhere + \labelsep, labelsep=\labelsep]
  \renewcommand\labelitemi{}                                            % Убираем маркеры
}{\end{itemize}}

% Альтернативное окружение для пояснений
\newenvironment{explanationx}{
  \noindent\tabularx{\textwidth}{@{}ll@{ --- } X }
}{\endtabularx}

% Игнорирование аргумента
\newcommand{\ignore}[2]{\hspace{0in}#2}

% Команда для приложений
\newcommand{\intro}[3]{
  \stepcounter{section}                                                 % Увеличиваем счетчик разделов
  \sectioncentered*{ПРИЛОЖЕНИЕ \MakeUppercase{#1}}                      % Заголовок
  \begin{center} 
    \bf{(#2)}\\                                                         % Обозначение (например, "А")
    \bf{#3}                                                             % Название
  \end{center}
  \markboth{\MakeUppercase{#1}}{}                                       % Колонтитулы
  \addcontentsline{toc}{section}{Приложение \MakeUppercase{#1} (#2) #3} % Запись в оглавление
}

% =====================================================================================================================
% Дополнительные настройки
% =====================================================================================================================
% Настройки подписей
\usepackage{caption}
\DeclareCaptionLabelFormat{stbfigure}{Рисунок \textit{#2}}          % Формат для рисунков
\DeclareCaptionLabelFormat{stbtable}{Таблица \textit{#2}}           % Формат для таблиц
\DeclareCaptionLabelFormat{stblisting}{Листинг \textit{#2}}         % Формат для листингов
\DeclareCaptionLabelSeparator{stb}{~--~}                            % Разделитель " - "
\captionsetup{
  labelsep=stb,                                                     % Применяем разделитель
  font={small}                                                      % Размер шрифта
}
\captionsetup[figure]{                                              % Настройки для рисунков
  labelformat=stbfigure, 
  justification=centering
}
\captionsetup[listing]{                                             % Настройки для листингов
  labelformat=stblisting,
  justification=centering
}
\captionsetup[table]{                                               % Настройки для таблиц
  labelformat=stbtable,
  justification=raggedright,                                        % Выравнивание по левому краю
  singlelinecheck=false                                             % Для многострочных подписей
}
\renewcommand{\thesubfigure}{\asbuk{subfigure}}                     % Нумерация подрисунков кириллицей

% Микротипографика
\usepackage{microtype}                                              % Улучшение читаемости текста

% Настройка сносок
\makeatletter 
\def\@makefnmark{\hbox{\@textsuperscript{\normalfont\@thefnmark)}}} % Стиль 1)
\makeatother
\usepackage[bottom]{footmisc}                                       % Сноски внизу страницы

% Разное
\usepackage{textcase}                                               % Верхний регистр
\usepackage{perpage}                                                % Нумерация сносок
\MakePerPage{footnote}                                              % Сноски на каждой странице
\usepackage{etoolbox}                                               % Утилиты для патчей
\usepackage{verbatim}                                               % Вставка кода
\usepackage{xcolor}                                                 % Цвета
\usepackage{minted}                                                 % Подсветка синтаксиса
\usepackage[normalem]{ulem}                                         % Подчеркивания
\DeclareRobustCommand{\x}[1]{\text{#1}}                             % Текст в математическом режиме
\renewcommand{\UrlFont}{\small\rmfamily\tt}                         % Стиль URL
\PassOptionsToPackage{hyphens}{url}                                 % Переносы в URL

% Отступ первой строки
\usepackage{indentfirst}                                            % Всегда отступ в первом абзаце
\newlength{\fivecharsapprox}                                        % Переменная для 5 символов
\setlength{\fivecharsapprox}{6ex}                                   % Эмпирическая ширина
\setlength{\parindent}{\fivecharsapprox}                            % Установка отступа