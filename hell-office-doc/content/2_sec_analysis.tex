\section{Анализ подходов к разработке системы управления и мониторинга информационной безопасности компании}
\label{sec:analysis}

В дaннoм рaзделе бyдет прoизвoдитьcя oбзoр пoдхoдoв для решения зaдaч, пocтaвленных в рaмкaх рaзрaбaтывaемoй системы, их преимyщеcтв и недocтaтков, a тaкже причины выбoрa кoнкретных пoдхoдов.

\subsection{Aнaлиз поставленной задачи в рамках предметной области}

Мониторинг и управление информационной безопасностью компании --- неотъемлемая часть рабочего процесса службы информационной безопасности компании. В связи со стремительным развитием информационных технологий, активно развивается преступность с использованием данных технологий. Мошенники используют различные алгоритмы и инструменты для получения доступа к материальным средствам и персональным данным (логин/пароль от различных персональных кабинетов пользователя, паспортные данные, идентификационные номера, данные платёжных карт и т.д.) с целью их похищения. Для решения проблемы компрометации персональных данных в компаниях могут использоваться различные программные решения:
\begin{itemize}
    \item антивирусные системы;
    \item фрод-мониторинг;
    \item межсетевые экраны (фаерволы);
    \item системы противодействия фишингу;
    \item системы администрирования доступов пользователей;
    \item системы защиты электронной почты;
    \item DLP-системы;
    \item СУБД.
\end{itemize}


Проблема мошенничества с использованием информационных технологий может быть решена внедрением и настройкой системы управления и мониторинга информационной безопасности компании. Такая система должна уметь:
\begin{itemize}
    \item своевременно идентифицировать локальные атаки и реагировать на них;
    \item в режиме реального времени проверять все транзакции, выявлять подозрительные и блокировать их;
    \item своевременно идентифицировать сетевые атаки и реагировать на них;
    \item в режиме реального времени производить поиск фишинговых ресурсов и информировать администраторов системы о таковых;
    \item инкапсулировать информацию;
    \item в режиме реального времени сканировать трафик электронной почты, выявлять подозрительные письма, блокировать их, и информировать об этом сотрудников службы информационной безопасности компании;
    \item предоставлять возможность удаленно администрировать рабочие машины;
    \item защищать, хранить, и структурировать внутреннюю информацию компании.
\end{itemize}


\subsection{Типовая структура системы управления и мониторинга информационной безопасности компании}

Типовая структура системы управления и мониторинга информационной безопасности компании состоит из элементов, представленных на рисунке \ref{structt}.

\begin{figure}[H]
  \centering
  \includegraphics[width=1\textwidth]{resources/1.jpg}
  \caption{Типовая структура системы управления и мониторинга информационной безопасности компании}
  \label{structt}
\end{figure}


Исходя из данных, представленных на рисунке \ref{struct}, сделан вывод, что APM администратора безопасности обменивается информацией с сервером мониторинга информационной безопасности, который, в свою очередь, собирает информацию из сети передачи данных, состоящей из следующих агентов мониторинга:
\begin{itemize}
    \item коммуникационное или сетевое оборудование (маршрутизаторы, коммутаторы, серверы доступа и другое);
    \item рабочие станции;
    \item серверы;
    \item средства защиты ИБ (системы выявления атак, межсетевые экраны, антивирусные системы и другие).
\end{itemize}

\subsubsection{Коммуникационное или сетевое оборудование}
Коммуникационное или сетевое оборудование --- устройства, необходимые для работы компьютерной сети, например: маршрутизатор, коммутатор, концентратор, коммутационная панель и др. Можно выделить активное и пассивное сетевое оборудование.

Активное оборудование — это оборудование, содержащее электронные схемы, получающее питание от электрической сети или других источников питания (батарейки, аккумулятора, солнечной панели, генератора и т. д.) и выполняющее функции преобразования, усиления сигналов и иные. Это определяет способность такого оборудования обрабатывать сигнал по специальным алгоритмам. В сетях происходит пакетная передача данных. Каждый сетевой пакет, помимо передаваемых данных, содержит, также, техническую информацию: сведения о его источнике, цели, целостности информации и другие, позволяющие доставить пакет по назначению. Активное сетевое оборудование не только улавливает и передает сигнал, но и обрабатывает эту техническую информацию, перенаправляя и распределяя поступающие потоки в соответствии со встроенными в память устройства алгоритмами. Эта «интеллектуальная» особенность, наряду с питанием от сети, является признаком активного оборудования. Например, в состав активного оборудования включаются следующие типы устройств:
\begin{enumerate}
    \item[1] Сетевой адаптер — плата, которая устанавливается в компьютер и обеспечивает его подсоединение к локальной вычислительной сети.
     \item[2] Коммутатор (свитч) (многопортовый мост) — устройство с несколькими (4-32) портами, обычно используемое для объединения нескольких рабочих групп ЛВС.
     \item[3] Маршрутизатор (роутер) — используется для объединения нескольких рабочих групп ЛВС, позволяет осуществлять фильтрацию сетевого трафика, разбирая сетевые (IP) адреса.
     \item[4] Ретранслятор — используется для создания усовершенствованной беспроводной сети с большей площадью покрытия и представляет собой альтернативу проводной сети. По умолчанию устройство работает в режиме усиления сигнала и выступает в роли ретрансляционной станции, которая улавливает радиосигнал от базового маршрутизатора сети или точки доступа и передает его на ранее недоступные участки.
     \item[5] Медиаконвертер — устройство, как правило, с двумя портами, обычно используемое для преобразования среды передачи данных (коаксиал-витая пара, витая пара-оптоволокно).
      \item[6] Сетевой трансивер — устройство, как правило, с двумя портами, обычно используемое для преобразования интерфейса передачи данных.
\end{enumerate}

Пассивное сетевое оборудование --- оборудование, не получающее питание от электрической сети или других источников питания (батарейка, аккумулятор, солнечная панель, генератор, др) и выполняющее функции распределения или снижения уровня сигналов. Например, кабельная система: кабель (коаксиальный и витая пара), вилка/розетка, коммутационная панель, симметрирующие устройство, преобразующие электрический сигнал из симметричного в несимметричный и наоборот для коаксиальных кабелей и т. д. Также, к пассивному оборудованию иногда относят оборудование трассы для кабелей: кабельные лотки, монтажные шкафы и стойки, телекоммуникационные шкафы.

\subsubsection{Рабочие станции}
Рабочие станции --- комплекс аппаратных и программных средств, предназначенных для решения определённого круга задач.
Рабочая станция как место работы специалиста представляет собой полноценный компьютер или компьютерный терминал, набор необходимого ПО, по необходимости дополняемые вспомогательным оборудованием: печатающее устройство, внешнее устройство хранения данных на магнитных и/или оптических носителях, сканер штрих-кода и прочим.
Также термином «рабочая станция» обозначают стационарный компьютер в составе локальной вычислительной сети по отношению к серверу. В локальных сетях компьютеры подразделяются на рабочие станции и серверы. На рабочих станциях пользователи решают прикладные задач: работают в базах данных, создают документы, делают расчёты, играют в компьютерные игры. Сервер обслуживает сеть и предоставляет собственные ресурсы всем узлам сети, в том числе и рабочим станциям.


\subsubsection{Серверы}
Сервер --- компьютер (или специальное компьютерное оборудование), выделенный из группы персональных компьютеров (или рабочих станций) для выполнения какой-либо сервисной задачи без непосредственного участия человека. Сервер и рабочая станция могут иметь одинаковую аппаратную конфигурацию, так как различаются лишь по участию в своей работе человека за консолью.
Некоторые сервисные задачи могут выполняться на рабочей станции параллельно с работой пользователя. Такую рабочую станцию условно называют невыделенным сервером.
Консоль и участие человека необходимы серверам только на стадии первичной настройки, при аппаратно-техническом обслуживании и управлении в нештатных ситуациях (штатно, большинство серверов управляются удалённо). Для нештатных ситуаций серверы обычно обеспечиваются одним консольным комплектом на группу серверов 

\subsubsection{Средства защиты ИБ}
Понятие «информационная безопасность» включает комплекс мер, направленных на предупреждение и устранение несанкционированного доступа, обработки, искажения, форматирования, анализа, несогласованного обновления, корректирования и уничтожения данных. Проще говоря, это комплекс действий, стандартов и технологий, необходимых для защиты конфиденциальных данных.
Цель защиты информации – сохранить данные и целостность системы, минимизировать потери в случае искажения информационных сведений. Сотрудники отделов информационной безопасности компании с помощью специального ПО могут отследить любое действие в корпоративной системе – создание, изменение, удаление, копирование и распространение важных файлов.
Для правильного внедрения средств обеспечения защиты конфиденциальной информации компании требуется соблюдать три основных принципа:
\begin{enumerate}
    \item[1] Целостность. Механизмы контроля должны работать в комплексе. Соблюдение принципа целостности обеспечивает отсутствие искажения данных и защиту от несанкционированных изменений.
     \item[2] Конфиденциальность. Введение мер контроля для создания адекватного уровня защиты данных, активов и информационной безопасности компании на различных этапах бизнес-операций, а также для устранения угрозы неправомерного доступа к корпоративной информации. Важно сохранять конфиденциальность при хранении информации, а также при передаче данных фирмам-посредникам, независимо от их степени важности.
     \item[3] Доступность. Обеспечение уполномоченных сотрудников нужной им информацией. Локальная сеть должна вести себя последовательно, чтобы в случае необходимости иметь доступ к цифровым данным.  Важным моментом является восстановление системы после любых сбоев, когда речь идет о доступе к данным. Метод восстановления не должен негативно влиять на функциональность предприятия. Без соблюдения вышеперечисленных принципов защита информации невозможна.
\end{enumerate}

На практике обеспечение информационной безопасности фирмы осуществляется с помощью следующих средств:
\begin{enumerate}
    \item[1] Моральные средства защиты.
Под моральными средствами подразумевают нормы поведения и правила работы с информационными активами, сложившиеся по мере распространения и внедрения электронной техники в различных отраслях государства и общества в целом. По факту это необязательные требования в отличие от законодательно утвержденных. Однако их нарушение приведет к потере репутации человека и организации.  
К морально-этическим средствам защиты информации в первую очередь стоит отнести честность и порядочность сотрудников. В каждой организации есть свой свод правил и предписаний, направленный на создание здорового морального климата в коллективе.  Механизмом обеспечения безопасности служит внутренний документ компании, учитывающий особенности деловых процессов и информационной структуры, а также устройство IT-системы.
     \item[2] Правовые средства защиты 
Они основываются на действующих в Российской Федерации законах, решениях и нормативных актах, устанавливающих правила обработки персональных данных, гарантирующих права и обязанности участникам при работе с информационными ресурсами в период их обработки и использования, а также возлагающих ответственность за нарушение этих постановлений, тем самым устраняя угрозу несогласованного использования конфиденциальной информации. Такие правовые методики используются в качестве профилактических и предупреждающих действий. В основном это организованные пояснительные беседы с персоналом предприятия, пользующимся корпоративными электронными устройствами. 
     \item[3] Организационные средства 
Это часть администрирования организации. Они регулируют функционирование системы обработки информации, работу штата организации и процесс взаимодействия работников с системой так, чтобы в большей степени устранить или предупредить угрозу информационной атаки либо уменьшить потери в случае их возникновения. Основной целью организационных мер является формирование внутренней политики в области сохранения в секрете конфиденциальных данных, включающей использование необходимых ресурсов и контроль за ними.
Внедрение политики конфиденциальности включает реализацию средств контроля и технических устройств, а также подбор персонала службы внутренней безопасности.  Возможны изменения в устройстве IT-системы, поэтому в реализации политики конфиденциальности должны участвовать системные администраторы и программисты. Персонал должен знать, почему проблемы сохранения коммерческой тайны столь важны. Все работники предприятия должны пройти обучение правилам работы с конфиденциальной информацией.
    \item[4] Физические средства защиты 
Это различные типы механических и электронно-механических устройств для создания физических препятствий при попытках нарушителей воздействовать на компоненты автоматизированной системы защиты информации. Это также технические устройства охранной сигнализации, связи и внешнего наблюдения. Средства физической безопасности направлены на защиту от стихийных бедствий, пандемий, военных действий и других внезапных происшествий.
    \item[5]Аппаратные средства защиты
Это электронные устройства, интегрированные в блоки автоматизированной системы или спроектированных как независимые устройства, контактирующие с этими блоками. Их задачей является внутренняя защита структурных компонентов ИТ-систем – процессоров, терминалов обслуживания, второстепенных устройств. Реализуется это с помощью метода управления доступом к ресурсам (идентификация, аутентификация, проверка полномочий субъектов системы, регистрация). 
    \item[6]Программные методы защиты 
Обеспечение сетевой безопасности осуществляется за счет специальных программ, которые защищают информационные ресурсы от несанкционированных действий. Благодаря универсальности, простоте пользования, способности к модифицированию программные способы защиты конфиденциальных данных являются наиболее популярными. Но это делает их уязвимыми элементами информационной системы предприятия. Сегодня создано большое количество антивирусных программ, брандмауэров, средств защиты от атак. Путем использования данных категорий программ, подходящим к используемым на предприятии информационным системам, создается комплексное обеспечение сетевой безопасности.
    \item[7]Технические способы защиты информационных данных 
Различные электронные устройства и специализированное оборудование, входящие в единый автоматизированный комплекс организации и выполняющие, как самостоятельные, так и комплексные функции сохранения персональных данных. К ним относятся персонализация, авторизация, верификация, ограничение доступа к активам пользователей, шифрование.
    \item[8]Криптографические методы защиты информации 
Этот метод основывается на способах кодировки и обеспечивает защиту конфиденциальной информации как с помощью программного обеспечения, так и аппаратными средствами защиты информации. Криптографический способ обеспечивает высокую степень эффективности СИЗ. Ее можно выразить в цифровом эквиваленте: среднее количество операций и время для разгадки ключей и расшифровки данных.  Для защиты текстов при передаче используются аппаратные методы кодировки, для обмена информацией между ПК локальной сети используются также программные методики. При сохранении информации на магнитных носителях используются программные методы шифровки. Однако у них есть некоторые недостатки: затраты времени и мощности процессоров для шифрования информации, трудности с расшифровкой, высокие требования к обеспечению секретности ключей (угроза открытых ключей от подмены).
Информация – это важнейшая часть современной действительности. Именно сейчас цифровые данные подвергаются растущему количеству угроз и нежелательных вторжений. DDoS-атаки, сетевой перехват данных, действие вирусного программного обеспечения и другие киберпреступления становятся более изощренными и набирают обороты. 
Поэтому следует как можно быстрее реализовать систему защиты информации, которая надежно защитит конфиденциальную корпоративную информацию. Вопрос безопасности информации полностью лежит на плечах руководства организацией. При выборе соответствующих средств для защиты информации следует принять во внимание область деятельности компании, ее размеры, техническое оснащение, а также компетенции персонала в сфере соблюдения режима конфиденциальности.
\end{enumerate}


\subsection{Обзор аналогов ключевого компонента системы управления и мониторинга информационной безопасности компании}
Ключевым компонентом в построении системы управления и мониторинга информационной безопасности компании является SIEM-система.
Понятие SIEM  (Security Information and Event Management) в наши дни достаточно размыто, можно представить, что это процесс, объединяющий сетевую активность в единый адресный набор данных. Сам термин был придуман Gartner в 2005 году, но с тех пор само понятие и все, что к нему относится, претерпело немало изменений. Первоначально аббревиатура представляла собой комбинацию двух терминов, обозначающих область применения ПО: SIM (Security Information Management) — управление информационной безопасностью и SEM (Security Event Management) — управление событиями безопасности.

Существует достаточно много SIEM решений на рынке ПО, рассмотрим несколько из них.

\textit{Security QRadar SIEM от IBM} регистрирует события с тысяч конечных устройств и приложений, распределенных в сети. Эта система выполняет мгновенную нормализацию и выявляет связь между действиями над необработанными данными, чтобы отличить реальные угрозы от ложных срабатываний.

Плюсы:
\begin{enumerate}
\item [1]обнаружение неправильного использования приложений, внутреннего мошенничества и современных небольших угроз, которые можно не заметить среди миллионов событий;
\item [2]выполнение мгновенной нормализации событий и сопоставление их с другими данными, полученными в результате обнаружения угроз, создания отчетов о соответствии требованиям и проведения аудита; 
\item [3]сокращение числа событий и потоков с миллиардов до небольшого количества реальных нарушений и определение приоритетов для них в соответствии с угрозой для бизнеса;
\item [4]использование опционального ПО IBM Security X-Force Threat Intelligence для определения действий, связанных с подозрительными IP-адресами, например, при подозрении во вредоносной активности.

\newpage
Минусы:
\end{enumerate}
\begin {enumerate}
\item[1]отсутствие мобильных версий;
\item[2]отсутствие интеграции с облачными сервисами;
\item[3]стоимость.
\end{enumerate}




\textit{KOMRAD от «НПО «Эшелон»} --- это SIEM-система, разработка российской компании ЗАО «НПО «Эшелон». Предназначена для оперативного оповещения и реагирования на внутренние и внешние угрозы безопасности автоматизированных систем, а также контроля выполнения требований по безопасности информации.

Особенности системы:
\begin{enumerate}
    \item [1]централизованный сбор и анализ данных журналов событий систем защиты информации, автоматизированных рабочих мест, серверов и сетевого оборудования;
    \item [2]удаленный контроль параметров конфигурации и работы отслеживаемых объектов;
    \item [3]оперативное оповещение и реагирование на внутренние и внешние угрозы безопасности автоматизированной системы;
    \item [4]контроль выполнения заданных требований по безопасности информации, сбор статистики и построение отчетов по защищенности;
    \item [5]поддержка технологии взаимодействия с источниками событий: Syslog, Syslog-ng, SNMPv2, SNMPv3, Opsec, HTTP, SQL, ODBC, WMI, FTP, SFTP, сокеты Unix/Linux, plain log, SSH, Rsync, Samba(NetBIOS), NFS, SDEE, RDEP, OPSEC, CPMI;
    \item [6]невысокая тоимость.
\end{enumerate}

Недостатки:
\begin{enumerate}
    \item [1]невозможность масштабирования решения и создания системы мониторинга информационной безопасности произвольного масштаба;
    \item [2]невозможность разбора событий в произвольном формате с помощью регулярных выражений (RE2) для подключения нестандартных источников событий информационной безопасности.
\end{enumerate}

\textit{Tibco Loglogic LogLogic SIEM} является модульной системой, состоящей из нескольких частей. LogLogic MX --- готовое решение для малого и среднего бизнеса. LogLogic ST --- долгосрочное хранение событий. LogLogic SEM --- корреляция и оповещение о событиях ИБ. LogLogic LX --- моментальный поиск событий. Database Security Manager ---  активный мониторинг и обнаружение уязвимостей баз данных.

Достоинства:

\begin{enumerate}

    \item [1]сбор событий с более чем 340 источников;
    \item [2]корреляция событий и оповещение в режиме реального времени;
    \item [3]моментальный поиск по данным за последние 90 дней;
    \item [4]хранение и поиск по данным за 10 лет.
\end{enumerate}

\subsection{Вывод}
В данном разделе был проведен анализ подходов к разработке системы управления и мониторинга информационной безопасности компании, была установлена структура системы, которая должна максимально удовлетворять потребности пользователя, а именно: своевременно идентифицировать локальные атаки и реагировать на них; в режиме реального времени проверять все транзакции, выявлять подозрительные и блокировать их; своевременно идентифицировать сетевые атаки и реагировать на них; в режиме реального времени производить поиск фишинговых ресурсов и информировать администраторов системы о таковых; инкапсулировать информацию; в режиме реального времени сканировать трафик электронной почты, выявлять подозрительные письма, блокировать их, и информировать об этом сотрудников службы информационной безопасности компании; предоставлять возможность удаленно администрировать рабочие
машины; защищать, хранить, и структурировать внутреннюю информацию
компании.