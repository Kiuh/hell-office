% !TEX root = ../hell-office.tex
\section{Проектирование системы управления и мониторинга информационной безопасности компании}
\label{sec:designing}

\subsection{Требования к системе в целом}
В рамках дипломного проекта поставлена цель разработать систему, которая должна удовлетворять запросам пользователей. Исходя из этого, выделен ряд требований к разрабатываемой системе:
\begin{itemize}
    \item система не должна оказывать влияние на функционирование систем, подлежащих мониторингу;
    \item система должна поддерживать возможность хранения событий за последние 365 дней;
    \item система должна поддерживать возможность быстрого доступа к событиям собранным в течение 90 дней;
    \item в системе должна быть предусмотрена возможность создания APM мониторинга с доступом с использованием веб-интерфейса;
    \item в системе должна быть предусмотрена возможность синхронизации времени, и маркировки всех событий по времени сервера управления данной системой;
    \item система должна быть гибкой к изменениям и добавлению новых возможностей и функционала.
\end{itemize}

\subsection{Требования к структуре и функционированию системы}
Система должна состоять из следующих структурных компонентов:
\begin{itemize}
    \item сервер управления (ArcSight Manager);
    \item сервер коннекторов (агентов) (ArcSight Smart Agent);
    \item APM управления (ArcSight Console).
\end{itemize}

Сервер управления системы управления и мониторинга информационной безопасности компании должен выполнять следующие функции: обработку поступающих с агентов событий аудита; управление коннекторами; управление учётными записями пользователей системы; хранение получаемых событий (при этом события хранятся как на встроенных в сервер управления дисках, так и на отдельном дисковом хранилище, подключённом к серверу консолидации событий).

Сервер коннекторов системы мониторинга и управления информационной безопасности компании является компонентом системы, на котором функционирует специализированное ПО сбора событий.

При помощи АРМ управления осуществляется управление и контроль системы.

\subsection{Требования к надежности системы}
Проектируемая и внедряемая системы мониторинга и управления информационной безопасности компании не должна уменьшать показатели надёжности информационных систем компании. Уровень надёжности разрабатываемой системы должен обеспечивать режим круглосуточного функционирования системы. Процедуры регламентного обслуживания с указанием времени их проведения должны быть отражены в рабочей документации. Должны быть предусмотрены процедуры резервного копирования собираемых данных аудита на внешние носители.

\subsection{Требования к защите от несанкционированного доступа в систему}
Каждый пользователь должен иметь уникальный идентификатор (логин) и пароль для входа в систему. Должна быть предусмотрена обязательная предварительная  регистрация пользователя уполномоченным лицом. Для доступа к журналам аудита в наблюдаемых системах должны создаваться технологические учётные записи, права доступа к данным, для которых присваиваются на основании принципа минимума привилегий. При использовании данных учëтных записей, система получает только возможность работы с журналом аудита наблюдаемых систем, без возможности чтения и/или изменения хранимых в системах основных данных. Функционал системы не должен предоставлять возможность получения доступа к данным автоматизированной системы компании, обрабатывая только события аудита, не затрагивая сами данные автоматизированной системы компании. Реализация механизмов идентификации и аутентификации должна удовлетворять следующим требованиям: доступ пользователю к APM управления системой должен предоставляться пользователю только после предъявления уникального персонифицированного идентификатора (имени) пользователя и проведения процедуры аутентификации на основе некоторой вводимой пользователем информации (пароль, ключи), при этом должна быть предусмотрена возможность смены такой информации по запросу пользователей периодической регламентной сменой пользовательских и административных паролей; должна быть обеспечена возможность определения авторства каждой операции проводимой в системе; при попытке подбора паролей при входе в APM системы (неправильный набор пароля три раза подряд), система должна блокировать работу пользователя. Система должна содержать средства управления правами доступа пользователей. Уровень доступа в системе должен быть определён по ролевому признаку (на основе дискретных правил). Должны быть доступны следующие роли: роль администратора системы; роль оператора системы. В системе должен быть предусмотрен механизм администрирования, предназначенный для выполнения функций по управлению системой. Выполнение административных функций в системе должно быть вынесено на прикладной уровень в виде отдельного АРМ или модуля администратора без возможности доступа к информации систем, подлежащим мониторингу. Должна быть исключена возможность удаления событий из системы, с использованием стандартных средств администрирования системы. Система регистрации событий должна протоколировать следующие действия: создание нового пользователя; назначение пользователю прав доступа к данным и функциям системы, а также изменение прав доступа; входы/выходы пользователей в/из системы; использование специальных полномочий (редактирование информации в базе данных, сторнирование проведенных операций и т.д.).

\subsection{Требования к функциональным характеристикам системы}
Проектируемая система должна осуществлять сбор событий из журналов событий информационной системы, подлежащих мониторингу, и сохранять результат обработки в централизованной базе данных. Также, система должна обеспечивать выполнение следующих требований: предоставлять функционал агрегации, нормализации и корреляции событий от источников. Параметры агрегации, нормализации и корреляции должны быть настраиваемыми; предоставлять функционал формирования отчётов по событиям, сохраненным в централизованной базе данных. Функционал формирования отчётов должен быть настраиваемым; учитывать подключение в дальнейшем объектов в других временных зонах и корректно обрабатывать переход на летнее/зимнее время; при подключении к источникам событий система должна подчиняться принципу использования наименьших привилегий.



\subsection{Требования к модернизации}

В системе должна быть предусмотрена возможность модернизации и обновления без потери и уменьшения производительности. Модель системы обязана иметь возможность адаптировать ее к изменяющимся условиям эксплуатации и обеспечивать возможность поэтапного обновления отдельных компонентов.


\subsection{Технические требования к реализации}

Реализация должна представлять собой систему управления и мониторинга информационной безопасности компании, предоставляющее функциональные возможности по обеспечению круглосуточной информационной безопасности компании.

Взаимодействие с системой должно осуществляться через глобальную сеть Интернет с использованием закрытых каналов передачи данных. 


\subsection{Вывод}
В данном разделе спроектирована модель системы управления и мониторинга информационной безопасности компании. Определены возможности системы, выделены основные сценарии использования.

Разработаны требования к системе в целом, дальнейший план модернизации и дополнению функциональных возможностей, и основные требования к реализации конечной системы. Также сформированы требования к структуре и функционированию системы управления и мониторинга информационной безопасности компании.

Также, структурированы понятия, связанные с разрабатываемой системой. Разобрана архитектура системы и связь между её компонентами.

Разработаны требования к отчётности, планировке и основные требования к реализации конечной системы.


