% !TEX root = ../hell-office.tex
\section{Разработка системы управления и мониторинга информационной безопасности компании}
\label{sec:development}
Структура системы управления и мониторинга информационной безопасности компании отображена на рисунке \ref{struct}

\begin{figure}[H]
  \centering
  \includegraphics[width=1\textwidth]{resources/2.jpg}
  \caption{Структурная схема системы управления и мониторинга информационной безопасности компании}
  \label{struct}
\end{figure}

\subsection{Используемые составляющие системы управления и мониторинга информационной безопасности компании}

\subsubsection{Microsoft Active Directory}
Службы Active Directory (AD) - решение от компании Microsoft позволяющее объединить различные объекты сети (компьютеры, сервера, принтера, различные сервисы) в единую систему. В данном случае AD выступают в роли каталога (базы данных), в котором хранится информация о пользователях, ПК, серверах, сетевых и периферийных устройствах.
Для реализации данного решения, необходим специальный сервер - контроллер домена. Именно он будет выполнять функции аутентификации пользователей и устройств в сети, а также выступать в качестве хранилища базы данных. При попытке использовать любой из объектов (ПК, сервер, принтер) сети, выполняется обращение к контроллеру домена, который либо разрешает это действие (есть необходимые права), либо блокирует его.

\textit{ЕДИНАЯ ТОЧКА АУТЕНТИФИКАЦИИ.}
Поскольку контроллер домена Active Directory хранит всю информацию об инфраструктуре и пользователях, вы легко можете использовать его для входа систему. Так, все данные пользователей (логины и пароли) хранятся в единой базе данных, что существенно упрощает работу с ними. При авторизации все компьютеры обращаются к этой базе данных, благодаря чему вносимые изменения будут применены ко всем компьютерам сети.  Также с помощью AD реализуются политики безопасности, благодаря котором можно ограничить (либо разрешить) доступ к определенным серверам.

\textit{УДОБНОЕ УПРАВЛЕНИЕ ПОЛИТИКАМИ.}
С помощью Active Directory можно поделить компьютеры на различные рабочие группы (организационные подразделения). Это существенно упрощает использование инфраструктуры в двух случаях:
\begin{enumerate}
    \item[1] Изменение существующих настроек группы. Поскольку настройки хранятся в единой базе данных, при их модификации, они будут применены для всех компьютеров, относящихся к этой группе.
    \item[2] Добавление нового пользователя. Он автоматически получает установленные для его группы настройки, что существенно ускоряет создание новой учетной записи.
\end{enumerate}

В зависимости от пользователя (учетной записи, которая используется) и его группы можно ввести ограничение на использование функционала операционной системы. Например, можно ограничить установку приложений всем кроме администраторов.

\textit{МОНИТОРИНГ СОБЫТИЙ В MICROSOFT ACTIVE DIRECTORY.}
К системе управления и мониторинга информационной безопасности компании должны быть подключены ресурсы Active Directory и серверы под управлением Microsoft Windows. Мониторинг событий Active Directory и серверов под управлением MS Windows необходимо вести удалённо, используя агент SmartConnector for Windows Event Log --- Unified, установленный на сервере агентов ArcSight.
Учётная запись, используемая для сбора событий безопасности с сервера агента ArcSight, не должна иметь административных полномочий в метакаталоге Active Directory и на контролируемых серверах.
Сбор событий безопасности должен производиться только из журнала безопасности серверов. Для этого пользователю, от имени которого осуществляется мониторинг событий безопасности с сервера агентов ArcSight, должны быть назначены полномочия для удалённого чтения журнала событий.
Система мониторинга и управления информационной безопасности компании должна считывать записи аудита Windows о следующих событиях безопасности Active Directory:
\begin{itemize}
    \item управление учётными записями пользователей, группами, ролями;
    \item попытки входа с неправильным паролем;
    \item попытка входа с заблокированным идентификатором;
    \item попытка доступа к неразрешённым правам и функциям;
    \item доступ и очистка журналов аудита;
    \item создание и удаление системных объектов;
    \item сигналы тревоги или сбоев в системе;
    \item доступ к файлам.
\end{itemize}

Должны быть настроены правила корреляции, позволяющие выявлять следующие нарушения безопасности (при условии предоставления штатной системой аудита достаточной информации):
\begin{itemize}
    \item подбор пароля;
    \item создание пользователя с последующим удалением в течении не более двух часов;
    \item назначение пользователю прав доступа (напрямую или включением в группу) с последующим их удалением в течении не более двух часов;
    \item добавление административных прав доступа пользователю;
    \item последовательный вход под разными идентификаторами (тремя и более) с одного АРМ в течении более двух часов;
    \item попытка параллельной аутентификации под одним идентификатором;
    \item отключения опций или функций аудита;
    \item доступ к файлам, содержащим данные платёжных карт.
\end{itemize}

Администратор безопасности должен иметь возможность задавать временной интервал и количество отказов, вызывающих создание инцидента.

\subsubsection{Система ORACLE Database 10g}
Oracle Database 10g --- база данных, разработанная специально для работы в сетях распределенных вычислений. Oracle Database 10g предназначена для эффективного развертывания на базе различных типов оборудования, от небольших серверов до Oracle Enterprise Grid мощных многопроцессорных серверных систем, от отдельных кластеров до корпоративных распределенных вычислительных систем.

Oracle Database 10g позволяет пользователям виртуализировать использование аппаратного обеспечения - серверов и систем хранения данных. Oracle Database 10g обладает технологиями, которые позволяют администраторам надежно хранить и быстро распределять и извлекать данные для пользователей и приложений, работающих в сетях Grid. Oracle Database 10g значительно повышает производительность обработки данных и включает в себя удобные средства администрирования.


Oracle Database 10g предоставляет возможность автоматической настройки и управления, которая делает ее использование простым и экономически выгодным. Ее уникальные возможности осуществлять управление всеми данными предприятия - от обычных операций с бизнес-информацией до динамического многомерного анализа данных (OLAP), операций с документами формата XML, управления распределенной/локальной информацией - делает ее идеальным выбором для выполнения приложений, обеспечивающих обработку оперативных транзакций, интеллектуальный анализ информации, хранение данных и управление информационным наполнением.
Некоторые ключевые возможности Oracle Database 10g:
\begin{itemize}
    \item[1] Real Application Cluster (RAC) обеспечивает работу одного экземпляра базы данных на нескольких узлах grid, позволяя управлять нагрузкой и гибко масштабировать систему в случае необходимости.
    \item[2] Automatic Storage Management (ASM) позволяет автоматически распределять данные между имеющимися ресурсами систем хранения данных, что повышает отказоустойчивость системы и снижает общую стоимость владения (TCO).
    \item[3] Производительность. Oracle Database 10g позволяет автоматически управлять уровнями сервиса и тиражировать эталонные конфигурации в рамках всей сети.
    \item[4] Простые средства разработки. Новый инструмент разработки приложений HTML DB позволит простым пользователям создавать эффективные приложения для работы с базами данных в короткие сроки.
    \item[5] Самоуправление. Специальные механизмы Oracle Database 10g позволяют самостоятельно перераспределять нагрузку на систему, оптимизировать и корректировать SQL-запросы, выявлять и прогнозировать ошибки.
    \item[6] Большие базы данных. Теперь максимальный размер экземпляра базы данных Oracle может достигать 8 экзабайт.
    \item[7] Недорогие серверные системы. Oracle Database 10g может использовать недорогие однопроцессорные компьютеры или модульные системы из "серверов-лезвий".
    \item[8] В новой версии базы данных реализована поддержка переносимых табличных пространств, система управления потоками данных Oracle Streams и модель распределенных SQL-запросов. Для переноса существующих баз данных в среду Grid в них не потребуется вносить изменений, что позволяет быстро начать использовать все преимущества Oracle Database 10g.
\end{itemize}

\textit{МАСШТАБ ИНФОРМАЦИОННОЙ СИСТЕМЫ И РЕДАКЦИИ СУБД ORACLE.}
Ядром СУБД является сервер базы данных, который поставляется в одной из четырех редакций (Oracle Database 10g Enterprise Edition, Oracle Database 10g Standard Edition, Oracle Database 10g Standard Edition One, Oracle Database 10g Personal Edition) в зависимости от масштаба информационной системы, в рамках которой предполагается его применение.
Для систем масштаба крупной организации предлагается продукт Oracle Database Enterprise Edition (корпоративная редакция), для которого имеется целый набор опций, архитектурно и функционально расширяющих возможности сервера. Продукт Oracle Database Standard Edition (стандартная редакция) ориентирован на организации среднего масштаба или подразделения в составе крупной организации. В рамках десятой версии СУБД Oracle стала доступной еще одна редакция - Standard Edition One, соответствующая функциональным возможностям Standard Edition, но доступная для лицензирования на компьютерах с числом процессоров не более двух. Персональная редакция (Personal Edition) предназначена, как следует из названия, для персонального применения. В стандартной и персональной редакциях основной акцент сделан на невысокую стоимость, простоту установки и сопровождения. При этом все варианты сервера Oracle имеют в своей основе один и тот же код и функционально идентичны, за исключением дополнительных модулей и опций, которые необходимы для специфических конфигураций. Основное преимущество такого подхода к построению СУБД - это идентичность кода для всех вариантов сервера баз данных. Для всех компьютерных платформ и архитектур существует единая СУБД Oracle, поставляемая в различных версиях, которая предоставляет одинаковую базовую функциональность вне зависимости от платформы.

\textit{ПОДДЕРЖИВАЕМЫЕ КОМПЬЮТЕРНЫЕ ПЛАТФОРМЫ И АРХИТЕКТУРЫ.}
Одной из основных характеристик СУБД Oracle является функционирование системы на большинстве платформ, и в том числе на больших ЭВМ, UNIX-серверах, персональных компьютерах и так далее. Другой важной характеристикой является поддержка Oracle всех возможных вариантов архитектур, в том числе симметричных многопроцессорных систем, кластеров, систем с массовым параллелизмом, архитектур мэйнфреймов. Очевидна значимость этих характеристик для современных организаций, где эксплуатируется множество компьютеров различных моделей и производителей. В таких условиях фактором успеха является максимально возможная типизация предлагаемых решений, ставящая своей целью существенное снижение стоимости владения программным обеспечением. Унификация систем управления базами данных - один из наиболее значимых шагов на пути достижения этой цели.

Поддержка Oracle большинства популярных компьютерных платформ и архитектур достигается за счет жесткой технологической схемы разработки кода СУБД. Разработку серверных продуктов выполняет единое подразделение корпорации Oracle, изменения вносятся централизовано, после этого все версии подвергаются тщательному тестированию в базовом варианте, а затем переносятся на все платформы, где также детально проверяются. Возможность переноса Oracle обеспечивается специфической структурой исходного программного кода сервера баз данных. Приблизительно 80\% программного кода Oracle - это программы на языке программирования C, который (с известными ограничениями) является платформонезависимым. Примерно 20\% кода, представляющее собой ядро СУБД, реализовано на машинно-зависимых языках, и эта часть кода перерабатывается для различных платформ. СУБД Oracle скрывает детали реализации механизмов управления данным на каждой из платформ, что дает основание говорить о практически полной унификации базового программного обеспечения. Дополнительно к этому, архитектура Oracle позволяет переносить прикладные системы, реализованные на одной платформе, на другие платформы без изменений как в структурах баз данных, так и кодов приложений.

\textit{КЛАССЫ ПРИЛОЖЕНИЙ.}
СУБД Oracle в одинаковой степени оптимизирована и для приложений оперативной обработки транзакций, и для аналитических приложений. На практике это означает, что один и тот же продукт (например, Oracle Database Enterprise Edition) можно с успехом использовать и как сервер оперативных баз данных, обрабатывающий интенсивный поток относительно простых и коротких транзакций, поступающих от множества пользователей, так и в качестве сервера хранилища данных, который позволяет концентрировать большие объемы данных и выполнять над ними сложные аналитические вычисления.

\textit{ШИРОКИЙ СПЕКТР ТИПОВ ДАННЫХ.}
Oracle опирается на стандарт SQL-3, позволяющий описывать определения новых типов объектов, состоящих из атрибутов (скалярных - то есть других типов, множеств объектов, ссылок на объекты) и обладающих ассоциированными с ним методами. Любая колонка таблицы может содержать данные базовых или сложных типов, поддерживаются также вложенные таблицы и массивы объектов переменной длины.
Одна из отличительных особенностей Oracle - возможность хранения и обработки различных предопределенных типов данных. Данная функциональность интегрирована в ядро СУБД и поддерживается модулем interMedia в составе Oracle Database. Он обеспечивает работу с текстовыми документами, включая различные виды поиска, в том числе контекстного; работу с графическими образами более 20-ти форматов; работу с аудио- и видео информацией. СУБД Oracle не просто предоставляет расширенный набор встроенных типов данных, но и позволяет конструировать новые типы данных со спецификацией методов доступа к ним. Это означает, что разработчики получают в руки не просто систему для хранения и обработки атрибутивных данных в виде таблиц, а инструмент, позволяющий строить структурированные типы данных, непосредственно отображающие сущности предметной области.

\textit{КОМПОНЕНТЫ И МОДУЛИ ORACLE DATABASE.}
Модуль interMedia обеспечивает поддержку всех типов данных, в том числе выполнение операций поиска по большим текстовым документам различных форматов.

Компонент Oracle Enterprise Manager представляет собой универсальное средство администрирования баз данных, снабженное удобным графическим интерфейсом и позволяющее администратору баз данных выполнять широкий спектр операций над множеством баз данных Oracle, включая создание, модификацию и удаление любых объектов внутри каждой из них.

Модуль Distribution Option позволяет эффективно работать с распределенными базами данных и обеспечивает двухфазную фиксацию транзакций к нескольким базам данных.

Модуль Advanced Replication Option позволяет выполнять репликацию данных в широком диапазоне возможностей, включая синхронную, асинхронную, каскадную и другие типы репликации.

Начиная с версии 8, СУБД Oracle является объектно-реляционной системой. Модуль Objects Option поддерживает объектно-ориентированные возможности: объектные типы, коллекции, массивы, вложенные таблицы, ссылки на объекты и большие бинарные объекты (BLOB).

За счет включения в сервер Oracle модуля 64 Bit Option, Oracle Database работает не только на 32-разрядных, но и на 64-разрядных компьютерах, что существенно расширяет его возможности как по скорости обработки данных, так и по объемам обрабатываемых данных.

Oracle Advanced Queuing (AQ) - встроенный в Oracle Database механизм хранения и обработки очередей сообщений. Компонент AQ относится к классу MOM (Message Oriented Middleware). Наличие такого компонента позволяет построить на базе сервера полнофункциональную инфраструктуру для обработки сообщений и исключает необходимость приобретения для этой цели дополнительных средств третьих фирм (таких как IBM MQ Series), обеспечивая, в то же время, связь с ними в неоднородных средах за счет продукта Oracle Messaging Gateway. AQ обеспечивает асинхронный режим обмена сообщениями между приложениями. AQ предлагает два режима рассылки сообщений: одна точка ко многим (point-to-multipoint) и публикация-подписка (publish/subscribe). AQ позволяет задавать приоритеты сообщений, задавать порядок сообщений в очереди (FIFO или на основе приоритета), группировать сообщения, определять правила доставки и время жизни сообщения, автоматически преобразовывать формат сообщения, получать по электронной почте асинхронные уведомления о прибытии интересующего сообщения, передавать сообщения через HTTP(S). Начиная с версии Oracle8i в состав сервера (во все редакции) включена виртуальная Java-машина (JServer Enterprise Edition).

Oracle Database снабжен всеми необходимыми средствами для подключения клиентских рабочих мест по протоколу Net8 (модуль Networking Kit), для обеспечения работы клиентов по технологии OLE (модуль Objects for OLE), набором ODBC-драйверов (ODBC Driver) и библиотеками для разработки программ на языках третьего уровня, использующих для доступа к базе данных Oracle Call Level Interface (OCI). Oracle Call Interface поддерживает разработку программ с применением вызовов низкоуровневых функций для доступа к базам данных. Это позволяет создавать эффективные программы, требующие минимальных ресурсов. Возможность разработки оптимизированных по скорости и используемой памяти приложений достигается за счет использования вызовов функций, которые предоставляют полный контроль за выполнением операторов SQL и PL/SQL.

Компонент Oracle Objects for OLE предоставляет возможность доступа к базам данных Oracle приложений, разработанных на C++, Microsoft Visual Basic, OLE 2.0. Полная поддержка языка макроопределений в Visual Basic позволяет получать данные из баз данных Oracle непосредственно в электронных таблицах Microsoft Excel.

\textit{ORACLE WORKFLOW.}
Oracle Workflow - это средство для автоматизации стандартных бизнес-процедур организации, ориентированное на разработчиков корпоративных приложений, основанных на технологиях Oracle.

Oracle Workflow предлагает инфраструктуру и средство проектирования (Workflow Builder) для автоматизации прохождения информации произвольного типа, формализации сложных бизнес-правил и включения пользователя в процесс принятия решения. Разработка приложений для управления потоками работ начинается с проектирования алгоритма процесса в графической среде Workflow Builder. Процесс состоит как из стандартных действий, таких как точки входа, выхода, ветвления, уведомления, вложенного процесса, так и действий, специфических для конкретного приложения, функциональность которых реализуется разработчиками. После того, как описания процессов сохранены в репозитории, они могут быть использованы приложениями через программный интерфейс. Дополнительные возможности включают рассылку почтовых уведомлений о результатах работы процесса и предоставление форм интерактивного взаимодействия пользователей с автоматизированным процессом, например, для получения подтверждений или контроля исполнения поручений.

\textit{ORACLE LITE.}
Oracle Database Lite (ODL) - программный продукт для создания инфраструктуры систем мобильных приложений. В состав продукта входит все необходимое для разработки, установки и управления приложениями для мобильных устройств на всех популярных сейчас ОС: Linux, Unix, Palm OS, Microsoft Windows CE/PPC, и Microsoft Windows NT/2000/XP. Основная задача предлагаемой инфраструктуры - обеспечение надежной и безопасной синхронизации данных между корпоративной базой данных Oracle Database и мобильными клиентами. После первого сеанса синхронизации пользователи, работая на компьютерах, где не было установлено никакого специального программного обеспечения, получают работающие приложения и базу данных ODL с актуальными корпоративными данными. При следующих сеансах связи пользователям передается только измененная информация. ODL - небольшая, но полнофункциональная реляционная база данных, специально спроектированная для работы на мобильных устройствах, в которой полностью реализованы механизмы транзакций, ссылочной целостности и спецификации языка SQL.

Бизнес-логика - хранимые процедуры и триггеры - разрабатывается на Java. Mobile Server - это расширение Oracle AS 10g, этот компонент обеспечивает взаимодействие мобильных приложений с Oracle Datаbase 10g или с различными Интернет-приложениями. При синхронизации данных, в случае разрыва соединения, передача информации на мобильные устройства возобновится после восстановления связи именно с той точки, где она прервалась. Применение Mobile Server обеспечивает гарантированную доставку данных. Информация, которая передается по сети и хранится в базе данных, может быть зашифрована по алгоритмам FIPS-140, удовлетворяющим стандартам AES. Синхронизация данных между базой данных Oracle Lite 10g и Oracle Database осуществляется по протоколам TCP/IP, HTTP, CDPD, 802.11b Wireless LAN, PPP, GPRS, HotSync, ActiveSinc. Программный интерфейс Open Transport API дает возможность использовать любой беспроводной транспортный протокол для синхронизации. Мобильные приложения разрабатываются с помощью Mobile Development Kit на языках программирования C, C ++, Java, Visual Basic, с использованием ActiveX Data Objects (ADO), в инструментальных средах Oracle JDeveloper 10g, Microsoft Visual Studio.Net 2003, Microsoft EVT 3.0, Borland Delphi, Sybase Power Builder, Metroworks CodeWarrior 8+, Rrapid Software Formation. Приложения, работающие на мобильных устройствах, имеют доступ к Oracle Lite 10g через различные программные интерфейсы (JDBC, ODBC, ADOCE, ADO.Net, SODA Stateless Object Database Access).

Уникальная опция ODL - Web-to-Go дает возможность приложениям, работающим через Web-навигатор, переключаться с режима прямого соединения на режим автономной работы. Пользователь в таком случае, синхронизировав локальные данные с информацией на корпоративном сервере, продолжает работать и при разрыве соединения.

\textit{МОНИТОРИНГ СОБЫТИЙ В СИСТЕМЕ ORACLE 10G.}
К системе управления и мониторинга информационной безопасности компании необходимо подключить СУБД Oracle. Мониторинг событий данной СУБД необходимо осуществлять удалённо, используя агент SmartConnector for Oracle Audit DB, установленный на сервере агентов ArcSight.
Система управления и мониторинга информационной безопасности компании должна считывать записи аудита Oracle о следующих событиях безопасности СУБД Oracle:
\begin{itemize}
    \item управление учётными записями пользователей, группами, ролями;
    \item попытки входа с неправильным паролем;
    \item попытка входа с заблокированным идентификатором;
    \item попытка доступа к объектам СУБД.
\end{itemize}

Должны быть настроены правила корреляции, позволяющие выявлять следующие нарушения безопасности (при условии предоставления штатной системой аудита достаточной информации):
\begin{itemize}
    \item подбор пароля;
    \item создание пользователя с последующим удалением в течении не более двух часов;
    \item назначение пользователю прав доступа (напрямую или включением в группу) с последующим их удалением в течении не более двух часов;
    \item добавление административных прав доступа пользователю;
    \item последовательный вход под разными идентификаторами (тремя и более) с одного АРМ в течении более двух часов;
    \item отключения опций или функций аудита;
    \item доступ к таблицам, содержащим данные платёжных карт.
\end{itemize}

Администратор безопасности должен иметь возможность задавать временной интервал и количество отказов.

\subsubsection{Система антивирусной защиты Symantec Endpoint Protection}
Symantec Endpoint Protection — это программное обеспечение, для эффективной защиты конечных точек IT-инфраструктуры компаний. Данное решение обеспечивает усиленную защиту, предотвращая информационные атаки на физические и виртуальные среды. На рисунке \ref{Symantec} представлен интерфейс системы Symantec Endpoint Protection.

\begin{figure}[H]
  \centering
  \includegraphics[width=1\textwidth]{resources/3.jpg}
  \caption{Интерфейс системы Symantec Endpoint Protection}
  \label{Symantec}
\end{figure}

Прозрачное внедрение необходимых инструментов безопасности в один агент с единой консолью управления, добавляет Symantec Endpoint Protection полезные функции, которые абсолютно не влияют на производительность системы. Использование сети Global Intelligence Network позволяет оперативно получать информацию о новых угрозах и реагировать на них в автоматическом режиме.

\textit{Основные функции Symantec Endpoint Protection:}
\begin{itemize}
    \item[1] Эффективная защита от вирусов и шпионских программ. Решение обеспечивает эффективную защиту от вирусов, руткитов, червей, ботов, троянских и шпионских программ, а также, от новых угроз.
    \item[2] Превентивное определение угроз. Благодаря технологиям Insight и SONAR, программное обеспечение распознает новые и быстро меняющиеся программы, содержащие вредоносный код, а также, ранее неизвестные угрозы и блокирует их работу.
    \item[3] Управление на основе интеллекта. Автоматизация процессов и централизованное управление предоставляют достоверные сведения об угрозах и мгновенно реагируют на них.
    \item[4] Новейшая защита от сетевых угроз. Функция блокировки общих точек уязвимости и защита браузера обеспечивают эффективную защиту от сетевых атак и несанкционированной загрузки приложений, межсетевой экран выполняет свою работу на основании установленных правил.
\end{itemize}

\textit{Основные возможности Symantec Endpoint Protection:}
\begin{itemize}
    \item[1] Технология SONAR 3 анализирует запущенные программы, определяет и блокирует вредоносный код, как в известных, так и новых, ранее неизвестных угрозах в режиме реального времени.
    \item[2] Решение Symantec Insight позволяет классифицировать файлы на безопасные и подверженные угрозам и более точно обнаруживать программы, содержащие вредоносный код на уровне персональных компьютеров, ноутбуков, которые работают на платформах Windows и Mac, а также, серверов и шлюзов. Данная технология является нечто большим, нежели обычный антивирус.
    \item[3] Почти 100\% выявление спама и предотвращение утечки данных, благодаря расширенной фильтрации содержимого файлов, которая позволяет определять и блокировать перемещение важной информации по электронной почте и посредством мгновенных сообщений.
    \item[4] Более точный анализ, базирующийся на глобальной мировой гражданской сети анализа угроз, обеспечивает четкое представление локальной и сетевой ситуации с угрозами.
    \item[5] Системы защиты для виртуальных сред осуществляют надежную защиту виртуальной инфраструктуры, выявляют виртуальные клиенты и управляют ими в автоматическом режиме.
    \item[6] Защита веб-шлюзов от различных сетевых угроз.
\end{itemize}

\textit{МОНИТОРИНГ СОБЫТИЙ В СИСТЕМЕ АНТИВИРУСНОЙ ЗАЩИТЫ SYMANTEC ENDPOINT PROTECTION.}
К системе управления и мониторинга информационной безопасности компании необходимо подключить сервер антивирусной защиты Seserver. Мониторинг событий в системе антивирусной защиты Symantec Endpoint Protection выполняется удалённо, с использованием агента Smart Connector for Symantec Endpoint Protection DB, установленного на сервере агентов ArcSight.

Система управления и мониторинга информационной безопасности компании должна фиксировать следующие события безопасности системы Symantec Endpoint Protection:
\begin{itemize}
    \item события обнаружения вирусов;
    \item события обновлений антивирусных баз;
    \item события запуска/остановки клиентских модулей;
    \item события изменения конфигурации;
    \item обнаружение рисков безопасности;
    \item журналирование сетевого трафика (при наличии соответствующих политик в ПО Symantec Endpoint Protection);
    \item обнаружение вредоносного ПО.
\end{itemize}

Должны быть настроены правила корреляции, позволяющие выявлять следующие нарушения безопасности: обнаружение вирусных эпидемий и выявление нарушений политики антивирусной защиты.



\subsubsection{Система антивирусной защиты Kaspersky Business Security}
В связи со стремительным развитием информационных технологий Все больше коммерческих операций выполняется в электронной форме, поэтому необходимо следить за безопасностью каждого устройства, находящегося в сети. Программное решение Kaspersky Business Security объединяет в себе многоуровневые технологии с гибким управлением в облаке и централизованными средствами контроля программ, веб-контроля и контроля устройств для решения вышеуказанной задачи. Также, данное ПО решает следующие задачи:
\begin{itemize}
    \item[1] защита от новейших угроз, в том числе от безфайловых вирусов;
    \item[2] укрепление безопасности рабочих мест и снижение уязвимости к кибератакам;
    \item[3] повышение производительности и защита сотрудников с помощью инструментов контроля;
    \item[4] защита серверов и рабочих мест без ущерба для производительности;
    \item[5] защита различных платформ – Windows, Mac, Linux, iOS и Android;
    \item[6] простое управление безопасностью из единой консоли.
\end{itemize}

\newpage
\textit{МОНИТОРИНГ СОБЫТИЙ В СИСТЕМЕ АНТИВИРУСНОЙ ЗАЩИТЫ KASPERSKY BUSINESS SECURITY.}
К системе управления и мониторинга информационной безопасности компании должны быть подключены серверы антивирусной защиты Kaspersky Business Security. Мониторинг событий в системе антивирусной зашиты Kaspersky Business Security выполняется удалённо, с использованием агента FlexConnector для KAV 8, установленного на сервере агентов ArcSight.
Система управления и мониторинга информационной безопасности компании должна фиксировать следующие события безопасности системы Kaspersky Business Security:
\begin{itemize}
    \item события обнаружения вирусов;
    \item события обновлений антивирусных баз;
    \item события запуска/остановки клиентских модулей;
    \item события изменения конфигурации.
\end{itemize}

Должны быть настроены правила корреляции, позволяющие выявлять следующие нарушения безопасности (при условии предоставления штатной системой аудита достаточной информации): обнаружение вирусных эпидемий и выявление нарушений политики антивирусной защиты.

\subsubsection{Сервера MS Forefront UAG}
Microsoft Forefront Unified Access Gateway (UAG) — это программный пакет, обеспечивающий безопасный удаленный доступ к корпоративным сетям для удаленных сотрудников и деловых партнеров. Его услуги включают обратный прокси-сервер, виртуальную частную сеть (VPN), DirectAccess и службы удаленных рабочих столов. UAG является частью предложения Microsoft Forefront. Microsoft прекратила выпуск продукта в 2014 году, хотя функция прокси веб-приложения в Windows Server 2012 R2 и более поздних версиях предлагает некоторые из своих функций.

\textit{МОНИТОРИНГ СОБЫТИЙ В СЕРВЕРАХ MS FOREFRONT UAG.}
К системе управления и мониторинга информационной безопасности компании должны быть подключены серверы MS Forefront UAG. Мониторинг событий выполняется удалённо, с использованием агента FlexConnector для MS Forefront UAG, установленного на сервере агентов ArcSight.
Система управления и мониторинга информационной безопасности компании должна фиксировать следующие события прокси-серверов MS Forefront UAG: фильтрация сетевых пакетов и доступ к сети Интернет.

Должны быть настроены компоненты системы управления и мониторинга информационной безопасности компании, позволяющие выявлять следующие показателя (при условии предоставления штатной системой аудита достаточной информации):
\begin{itemize}
    \newpage
    \item основные получатели трафика;
    \item основные отправители трафика;
    \item основные запрошенные URL;
    \item основные пропущенные/заблокированные запросы;
    \item доступ к серверам, содержащим данные платёжных карт.
\end{itemize}

\subsubsection{Прокси-сервер Squid}
В корпоративных сетях довольно обычна ситуация, когда, с одной стороны, множество пользователей на разных компьютерах пользуются ресурсами сети Интернет, при этом обеспечить надлежащий уровень безопасности на этих компьютерах одновременно довольно сложно. Ещё сложнее заставить пользователей соблюдать некий «корпоративный стандарт», ограничивающий возможности использования Интернет (например, запретить использование определённого типа ресурсов или закрыть доступ к некоторым адресам).

Простейшим решением будет разрешить только определённые методы доступа к Интернет (например, по протоколам HTTP и FTP) и определить права доступа абонентов на одном сервере, а самим абонентам разрешить только обращение к этому серверу по специальному прокси-протоколу (поддерживается всеми современными браузерами). Сервер же, после определения прав доступа, будет транслировать (проксировать) приходящие на него HTTP-запросы, направляя их адресату. Допустим, несколько пользователей с нескольких компьютеров внутренней сети просматривают некоторый сайт. С точки зрения этого сайта их активность представляется потоком запросов от одного и того же компьютера.

\textit{МОНИТОРИНГ СОБЫТИЙ В ПРОКСИ-СЕРВЕРЕ SQUID.}
К системе управления и мониторинга информационной безопасности компании необходимо подключить сервер Squid. Мониторинг событий выполняется удалённо, с использованием агента FlexConnector для Squid SAMS, установленного на сервере агентов ArcSight.
Система управления и мониторинга информационной безопасности компании должна фиксировать следующие события прокси-серверов Squid: доступ к сети Интернет.

Должны быть настроены компоненты системы управления и мониторинга информационной безопасности компании, позволяющие выявлять следующие показателя (при условии предоставления штатной системой аудита достаточной информации):
\begin{itemize}
    \item основные пользователи-генераторы трафика;
    \item основные сайты-генераторы трафика;
    \item основные запрошенные URL;
    \item объём полученной и отправленной информации.
\end{itemize}


\subsubsection{СУБД MS SQL Server}
Microsoft SQL Server --- это ПО, которое работает на различных устройствах, таких как: персональный компьютер, ноутбук, виртуальная машина, сервер и даже "облако". К MS SQL Server можно подключаться локально или по сети, отправить команду по специальному протоколу TDS и, соответственно, получить ответ. Также, данная СУБД позволяет хранить и обрабатывать большой объём информации. Все что она делает это открывает сетевой порт, принимает запрос пользователя и даёт на данный запрос ответ. При работе в локальной сети, необходимо установить ПО MS SQL Server на каждую рабочую машину.

Центральным аспектом в MS SQL Server, как и в любой СУБД, является база данных. База данных представляет хранилище данных, организованных определенным способом. Нередко физически база данных представляет файл на жестком диске, хотя такое соответствие необязательно. Для хранения и администрирования баз данных применяются системы управления базами данных или СУБД. И как раз MS SQL Server является одной из таких СУБД.

Также, бывают различные виды SQL-сервером. На сегодняшний день наиболее популярными являются следующие СУБД:
\begin{itemize}
    \item[1] MS SQL server --- многопользовательский программный продукт, разработанный компанией Microsoft, обладающий высокой производительностью и отказоустойчивостью, тесно интегрированный с ОС Windows. Этот сервер поддерживает удаленные подключения, работает с многими популярными типами данных, дает возможность создавать триггеры и хранимые данные, имеет практичные и удобные утилиты для настройки.
    \item[2] Oracle Database server --- СУБД, предназначенная для создания, консолидации и управления базами данных в облачной среде. Используя этот сервер, можно как автоматизировать обычные бизнес-операции, так и выполнять динамический многомерный анализ данных, проводить операции с документами xml-формата и управлять разделенной и локальной информацией.
    \item[3] IBM DB2 - семейство СУБД для работы с реляционными базами данных, признанное самым производительным, имеющим высокие технические показатели и возможности масштабирования. SQL-серверы этой группы характеризуются мультиплатформенностью, способностью к мгновенному созданию резервных копий и восстановлению БД, реорганизации таблиц в онлайн-режиме, разбиению баз данных, определению пользователями новых типов данных.
   \item[4] MySQL - СУБД, разработанная и поддерживаемая компанией Oracle. В основном она используется локальными или удаленными клиентами, позволяя им работать с таблицами разных типов, поддерживающих полнотекстовый поиск или выполняющих транзакции на уровне отдельных записей.
   \item[5] PostgreSQL - СУБД с открытым исходным кодом, работающая с объектно-реляционными (поддерживающими пользовательские объекты) базами данных. Также PostgreSQL предназначена для создания, хранения и извлечения сложных структур данных. Она поддерживает самые различные типы данных (среди них - числовые, текстовые, булевы, денежные, бинарные данные, сетевые адреса, xml и другие).
\end{itemize}


При работе с MS SQL Server был выявлен ряд положительных свойств системы, таких как: производительность, так как SQL Server работает очень быстро; надежность и безопасность, так какSQL Server предоставляет шифрование данных; простота, так как с данной СУБД относительно легко работать и вести администрирование.

\textit{МОНИТОРИНГ СОБЫТИЙ В СУБД MS SQL SERVER.}
К системе управления и мониторинга информационной безопасности компании необходимо подключить серверы СУБД MS SQL Server. Мониторинг событий безопасности выполняется удалённо, с использованием агента SmartConnector for Microsoft SQL Server Multiple Instance Audit DB, установленного на сервере агентов ArcSight.
Для выявления событий, свидетельствующих о нарушении ИБ, должен быть включён аудит событий в контролируемых базах MS SQL Server. Система управления и мониторинга информационной безопасности компании должна считывать записи аудита MS SQL Server о следующих события безопасности:
\begin{itemize}
    \item вход пользователя в СУБД;
    \item выход пользователя из СУБД;
    \item ошибка аутентификации при подключении к СУБД;
    \item создание пользователя СУБД;
    \item использование привилегий администратора при работе с СУБД;
    \item доступ к объектам СУБД.
\end{itemize}

На рисунке \ref{source} продемонстрирована схема источников данных в системе мониторинга и управления информационной безопасностью банка.
\begin{landscape}
\begin{figure}[H]
  \centering
  \includegraphics[width=1.3\textwidth]{resources/4.jpg}
  \caption{Схема источников данных}
  \label{source}
\end{figure}
\end{landscape}

Должны быть настроены правила корреляции, позволяющие выявлять следующие нарушения безопасности (при условии предоставления штатной системой аудита достаточной информации):
\begin{itemize}
    \item подбор пароля;
    \item создание пользователя с последующим удалением в течении не более двух часов;
    \item добавление административных прав доступа пользователю;
    \item последовательный вход под разными идентификаторами (три и более) с одного АРМ в течении не более двух часов;
    \item доступ к таблицам, содержащим данные платёжных карт.
\end{itemize}

\subsubsection{Система OpenWAY}
Система OpenWAY используется для эмиссии и эквайринга карт, маршрутизации транзакций и омни-канальных платежей. С помощью данного ПО собирается, обрабатывается, и хранится информация о платежах. Также, на основании собранной информации можно сформировать отчёт.

\textit{МОНИТОРИНГ СОБЫТИЙ В СИСТЕМЕ OPENWAY.}
К системе управления и мониторинга информационной безопасности компании необходимо подключить системы OpenWAY. Мониторинг событий безопасности выполняется удалённо, используя агент FlexConnector for OpenWAY, установленный на сервере агентов ArcSight.
Система управления и мониторинга информационной безопасности компании должна считывать записи аудита системы OpenWAY о следующих событиях безопасности:
\begin{itemize}
    \item управление учётными записями пользователей, группами, ролями;
    \item выход пользователя из СУБД;
    \item использование привилегий администратора при работе с системой (доступ к журналам аудита, их очистка);
    \item попытка входа с неправильным паролем;
    \item попытка входа с заблокированным идентификатором;
    \item доступ к объектам.
\end{itemize}
Должны быть настроены правила корреляции, позволяющие выявлять следующие нарушения безопасности (при условии предоставления штатной системой аудита достаточной информации):
\begin{itemize}
    \item подбор пароля;
    \item создание пользователя с последующим удалением в течении не более двух часов;
    \item назначение пользователю прав доступа (напрямую или включением в группу) с последующим их удалением в течении не более двух часов;
    \item добавление административных прав пользователю;
    \item последовательный вход под разными идентификаторами (три и более) с одного АРМ в течении не более двух часов;
    \item отключение опций или функций аудита;
     \item доступ к объектам, содержащим данные платёжных карт.
\end{itemize}
Администратор безопасности должен иметь возможность задавать временной интервал и количество отказов.

\subsubsection{Операционная система Linux}
Операционная система Linux --- это семейство ОС, работающих на основе одноименного ядра, которое включает в себя множество различных дистрибутивов, как для общего пользования, так и для узкого применения. Над дистрибутивами операционной системы Linux работает большая децентрализованная команда активичтов, поэтому данная ОС распространяется абсолютно бесплатно и с открытым исходным кодом. На рисунке \ref{Linux}, рисунке \ref{Linux2} и рисунке \ref{Kali} представлены интерфейсы различных дистрибутивов операционной системы Linux.

\begin{figure}[H]
  \centering
  \includegraphics[width=1\textwidth]{resources/5.jpg}
  \caption{Интерфейс операционной системы Linux Ubuntu}
  \label{Linux}
\end{figure}

\begin{figure}[H]
  \centering
  \includegraphics[width=1\textwidth]{resources/6.png}
  \caption{Интерфейс операционной системы Linux Alpine}
  \label{Linux2}
\end{figure}
\begin{figure}[H]
  \centering
  \includegraphics[width=1\textwidth]{resources/7.jpg}
  \caption{Интерфейс операционной системы Kali Linux}
  \label{Kali}
\end{figure}

\textit{МОНИТОРИНГ СОБЫТИЙ В ОПЕРАЦИОННОЙ СИСТЕМЕ LINUX.}
К системе управления и мониторинга информационной безопасности компании необходимо подключить серверы под управлением ОС Linux. Мониторинг событий в ОС Linux необходимо осуществлять удалённо, используя агент ArcSight Syslog Daemon, установленный на сервере агентов ArcSight.
Система управления и мониторинга информационной безопасности компании должна фиксировать следующих событиях безопасности ОС Linux:
\begin{itemize}
    \item вход пользователя в систему;
    \item создание пользователя;
    \item использование привилегий администратора при работе с системой (доступ к журналам аудита, их очистка);
    \item изменение уровня привилегий пользователя;
    \item доступ к файлам.
\end{itemize}
Должны быть настроены правила корреляции, позволяющие выявлять следующие нарушения безопасности (при условии предоставления штатной системой аудита достаточной информации):
\begin{itemize}
    \item подбор пароля;
    \item создание пользователя с последующим удалением в течении не более двух часов;
    \item назначение пользователю прав доступа (напрямую или включением в группу) с последующим их удалением в течении не более двух часов;
    \item добавление административных прав пользователю;
    \item последовательный вход под разными идентификаторами (три и более) с одного АРМ в течении не более двух часов;
    \item отключение опций или функций аудита;
     \item доступ к объектам, содержащим данные платёжных карт.
\end{itemize}
Администратор безопасности должен иметь возможность задавать временной интервал и количество отказов.

\subsubsection{SIEM - система ArcSight ESM от Micro Focus}

Процесс мониторинга является одним из основных процессов в области информационной безопасности. Для достижения максимального эффекта мониторинг необходимо производить круглосуточно. Также, целью процесса является непрерывная регистрация, анализ и контроль событий безопасности. Под событием безопасности понимается событие, о котором просигнализировало некоторое средство защиты. Средством защиты выступает программно-аппаратный комплекс, утвержденный на предприятии. Сообщения об инциденте информационной безопасности поступают оператору безопасности, который оценивает данное сообщение и принимает решение о дальнейших действиях.

На рисунке \ref{ArcSight1}  представлен интерфейс сетевого монитора SIEM ArcSight.

\begin{figure}[H]
  \centering
  \includegraphics[width=1\textwidth]{resources/8.jpg}
  \caption{Интерфейс сетевого монитора SIEM ArcSight}
  \label{ArcSight1}
\end{figure}

На рисунке \ref{ArcSight2} представлен интерфейс менеджера событий SIEM ArcSight.


\begin{figure}[H]
  \centering
  \includegraphics[width=1\textwidth]{resources/9.jpg}
  \caption{Интерфейс менеджера событий SIEM ArcSight}
  \label{ArcSight2}
\end{figure}

Результаты процесса мониторинга передаются в процесс разрешения инцидентов для идентификации инцидентов ИБ и их обработки. Под инцидентом ИБ понимается событие безопасности, которое привело (либо может привести) к негативным последствиям. Также события безопасности являются исходными данными при анализе эффективности системы и могут помочь оценить соответствие функционирования системы требованиям ИБ.

На рисунке \ref{life_cycle} представлена схема жизненного цикла события в ArcSight.
\begin{figure}[H]
  \centering
  \includegraphics[width=0.8\textwidth]{resources/10.jpg}
  \caption{Cхема жизненного цикла события в ArcSight}
  \label{life_cycle}
\end{figure}

Задача регистрации подразумевает сбор событий безопасности от средств защиты и их независимое (вынесенное за пределы контролируемых средств защиты) хранения. Доступ к журналам регистрации осуществляется только администраторами безопасности.

Эффективная система мониторинга должна объединять события, собираемые со всех используемых защитных средств, в единую управляемую систему информационной безопасности, оставлять только те, в которых может содержаться полезная информация. Как показывает практика, из миллиона генерируемых записей интерес для администратора безопасности представляют сотни, если не десятки из них.

Вышеперечисленные проблемы способна решить система мониторинга, предоставляющая эффективные средства для анализа событий безопасности, а именно:
\begin{itemize}
    \item фильтрации данных;
    \item приоритезации данных;
    \item нормализации данных;
    \item агрегирования данных;
    \item корреляции данных.
\end{itemize}


На рисунке \ref{ArcSightMon} представлен интерфейс активного канала системы мониторинга SIEM ArcSight.

\begin{figure}[H]
  \centering
  \includegraphics[width=1\textwidth]{resources/11.png}
  \caption{Интерфейс активного канала системы мониторинга SIEM ArcSight}
  \label{ArcSightMon}
\end{figure}

Основной целью системы мониторинга событий информационной безопасности в автоматизированной системе является непрерывная регистрация, анализ и контроль событий безопасности.

Событие безопасности --- событие, зафиксированное некоторым средством.

Регистрация событий безопасности --- перемещение данных журналов из устройств безопасности или их систем управления в единую базу данных.

Анализ событий безопасности --- состоит из следующих составных частей: фильтрация данных, приоритезация данных, нормализация данных, агрегация данных, корреляция данных.

Контроль событий безопасности --- включает в себя отображение зафиксированных событий безопасности на единой консоли в режиме реального времени, оповещение администраторов безопасности о зафиксированных событиях и формирование отчётов.

В качестве системы, которая бы смогла реализовать рассмотренные выше задачи, рассматривается система корреляции событий ArcSight ESM, которая позволяет собирать и анализировать сообщения о нарушениях безопасности, поступающих от систем обнаружения вторжений, межсетевых экранов, операционных систем, приложений, антивирусных систем и т.д. Данная информация собирается в едином центре, обрабатывается и подвергается анализу в соответствии с заданными правилами по обработке событий, связанных с безопасностью. Результаты анализа в режиме реального времени предоставляются операторам, администраторам и руководящему составу компании в удобном виде для принятия решений по управлению сети.

Программный продукт ArcSight позволяет реализовать рассмотренные выше задачи. Данный продукт состоит из трёх логических уровней: уровень агентов, уровень серверных и дополнительных модулей и уровень взаимодействия с пользователем.

На рисунке \ref{ArcSight4} представлен интерфейс графической формы активного листа SIEM ArcSight. 

\begin{figure}[H]
  \centering
  \includegraphics[width=1\textwidth]{resources/12.jpg}
  \caption{Интерфейс графической формы активного листа SIEM ArcSight}
  \label{ArcSight4}
\end{figure}

Активный лист в SIEM ArcSight предназначен для отображения событий информационной безопасности в реальном времени. Также, активный лист обеспечивает возможность построения зависимостей между событиями с дальнейшим отображением полученной информации.

На рисунке \ref{alco} изображена блок-схема алгоритма реагирования на инциденты.
\begin{landscape}
\begin{figure}[H]
  \centering
  \includegraphics[width=1.5\textwidth]{resources/13.jpg}
  \caption{Блок-схема алгоритма реагирования на инцидент}
  \label{alco}
\end{figure}
\end{landscape}

В фазе нормализации и агрегирования события безопасности собираются со всех систем обнаружения вторжений: межсетевых экранов, операционных систем, приложений и антивирусных систем, и приводятся к единому формату сообщений в XML. Сформированные сообщения затем коррелируются между собой, используя мощные механизмы корреляции, основанные как на статистических методах корреляции, так и на задаваемых правилах. И, наконец, ArcSight ESM выдает полученные результаты на основанную на технологии Java, интуитивно понятную централизованную консоль, работающую в режиме реального времени.

ArcSight ESM помогает администраторам безопасности сфокусироваться на реальных угрозах безопасности, обеспечивая их средствами, позволяющими оперативно устранять угрозы безопасности сети. Также, данная SIEM-система поддерживает сбор данных с маршрутизаторов, систем предотвращения вторжений , и производит корреляцию этих данных с информацией, получаемой от межсетевых экранов, серверов, баз данных и многих приложений. В данный момент число типов поддерживаемых устройств --- более 200 (показатель больший, чем у Cisco MARS и SIMS вместе взятых.

Ниже приведен перечень основных механизмов функционирования системы ArcSight ESM с подробных описанием.
\begin{enumerate}
    \item [1] Фильтрация данных --- устранение избыточной информации, основываясь на критериях заданных администратором или определенных в системе.
    \item[2] Нормализация данных --- приведение данных от различных средств защиты к единому виду, единым показателям значимости события. Нормализация также подразумевает устранение избыточной информации, т.е. устранение повторяющихся данных о событии, полученных от разных средств защиты, исключение противоречивости в их хранении.
    \item[3] Агрегирование данных --- объединение однотипных событий в одно, другими словами это процесс группировки событий по одному или нескольким критериям, то есть последовательность однотипных событий заменяется одним, где в поле "количество" стоит количество агрегированных событий.
    
    Некоторые события, например сканирование портов, вызывают генерацию большого количества событий от одного или нескольких устройств. Реально происходит одно событие --- сканирование портов.
    
    Механизм агрегации позволяет сгруппировать серию событий и получить одно сообщение, что сильно упрощает анализ и экономит память. В системе возможна группировка по многим параметрам, среди которых:
    \begin{itemize}
        \item устройство, сгенерировавшее сообщение;
        \item адрес источника;
        \item адрес получателя;
        \item порт источника;
        \item порт получателя;
        \item внешний тип сообщения;
        \item внутренний тип сообщения;
        \item направление;
        \item процесс;
        \item имя пользователя.
    \end{itemize}
    
Агрегация может производиться на множестве внутренних сообщений, а также на множестве сообщений устройств, что повышает точность преобразования. Возможна настройка параметров агрегации (время и критерии сравнения) для каждого типа сообщений.

    \item[4] Корреляция данных --- способ выявления комплексных атак, т.е. атак, которые не могут быть обнаружены с использованием известных сигнатур IDS, так как фактически состоит из комплекса сигнатур, которые по отдельности не говорят о реальной атаке. Корреляция событий безопасности позволяет также проводить анализ событий на предмет выявления наличия комплекса событий, свидетельствующего об аномальном поведении, реализации угроз, к примеру, обнаружение подбора паролей.
    Возможны следующие варианты корреляции данных:
    \begin{itemize}
        \item корреляция с данными об операционной системе (например, обнаружение Unix-атаки направленной на ресурс под управлением ОС Windows не является угрозой и может быть проигнорировано администратором);
        \item корреляция атак и уязвимостей (если при проведении предварительного сканирования ресурсов было определено что определенный узел уязвим к определенному виду атак, то при обнаружении такой атаки направленной на данный ресурс событию будет присвоена высокая степень критичности, при определённой ранее неуязвимости узла --- низкая степень критичности);
        \item анализ шаблона атаки (позволяет сделать вывод о применении того или иного средства реализации атаки);
        \item сопоставление данных (несколько однотипных событий за заданный период времени объединяются в одно, например, подбор пароля).
    \end{itemize}
    
    В системе ArcSight реализовано два механизма корреляции --- статистический и корреляция, основанная на правилах.
    
    Статистическая корреляция --- встроенный эвристический механизм, позволяющий производить оценку рисков активов без использования предопределённых правил, основываясь только на данных, полученных от устройств. Администратор системы имеет возможность задать индивидуальные параметры ресурсов, связанные с анализом рисков (значимость, уязвимость, доступность пользователям). В любой момент времени можно сгенерировать отчёты, содержащие значение рисков ресурсов. Причину каждого значения риска можно определить, создав связанные детализированные отчёты.
    
    Корреляция, основанная на правилах --- механизм выявления события, соответствующего появлению последовательности событий, удовлетворяющей определенным правилам. В системе созданы правила для наиболее типичных атак, но имеется возможность создания собственных. 
    
   
    Результатом работы системы корреляции является вывод следующих сообщений:
    \begin{itemize}
        \item вероятно удачная атака (узел уязвим);
        \item вероятно неудачная атака (узел не уязвим);
        \item вероятно неудачная атака (блокированы некоторые пакеты, составляющие атаку);
        \item вероятно неудачная атака (атака не применима к данной операционной системе);
        \item неудачная атака (узел блокировал атаку);
        \item воздействие неизвестно (узел не сканировался);
        \item воздействие неизвестно (операционная система не определена);
        \item воздействие неизвестно (уязвимость не определена);
        \item воздействие неизвестно (корреляция не проводилась).
    \end{itemize}
    
    Также в результате работы системы корреляции возможно получение некоторого объединения событий:
    \begin{itemize}
        \item атака со скомпрометированного узла;
        \item распределенная DoS-атака;
        \item неудачная попытка входа на множество узлов;
        \item доступ к ресурсу со взломанного узла;
        \item удачная попытка входа на взломанный узел;
        \item попытка взлома просканированного узла;
        \item скоординированная атака;
        \item атака с одного узла на множество;
        \item атака через промежуточный узел.
    \end{itemize}
    
    \item[5] Приоритезация данных --- автоматическое присваивание событиям соответствующего уровня, исходя из заданных администратором или определенных в системе критериев.
    \item[6] Категоризация --- каждое внутреннее сообщение системы проходит сопоставление с предопределенными значениями и получает какой-либо статус. Например, категория Outcome, может принимать значение Success или Failure.
    Категоризация расширяет возможности системы генерации отчётов и представления данных на консоли.
    \item[7] Визуализация --- отображение консолидированных данных в реальном времени на консоли после после процессов нормализации, агрегации и корреляции.
\end{enumerate}

Система ArcSight ESM состоит из нескольких раздельно устанавливаемых компонентов, которые совместно обрабатывают события информационной безопасности. К основным компонентам системы мониторинга на базе ПО ArcSight ESM относятся:
\begin{itemize}
    \newpage
    \item агенты ArcSight SmartConnector;
    \item модуль ArcSight Manager;
    \item база данных ArcSight Database;
    \item консоль управления ArcSight Console;
    \item web интерфейс ArcSight Web.
\end{itemize}

\begin{figure}[H]
  \centering
  \includegraphics[width=1\textwidth]{resources/13.jpg}
  \caption{Схема коммуникации модулей ArcSight}
  \label{comm}
\end{figure}

Агенты ArcSight SmartConnectors осуществляют сбор и первичную обработку сообщний информационной безопасности, поступающих от различных источников. Агенты выполняют нормализацию, фильтрацию, агрегацию, первичную категоризацию сообщений и передачу обработанных сообщений в модуль управления ArcSight Manager.

Агенты могут устанавливаться на выделенный сервер агентов и собирать события ИБ удаленно, или непосредственно на контролируемые системы.

При установке нового агента, он регистрируется в модуле управления и всё дальнейшее взаимодействие с модулем управления осуществляется по зашифрованному каналу. Агент передает принятые и обработанные сообщения ИБ модулю управления пакетами по 100 событий или каждую секунду, что наступает раньше.

Для получения событий ИБ с нестандартных или не поддерживаемых ArcSight ESM систем используются программируемые агенты ArcSight FlexConnector.

Модуль управления (ArcSight Manager) сохраняет обработанную информацию в базе данных (ArcSight DataBase) и обрабатывает события ИБ с помощью механизма корреляции, который оценивает события с учётом сетевой модели и информации о уязвимостях, оценивая тем самым угрозы в реальном времени. Для этого модуль управления содержит большое количество предварительных настроек фильтров, правил, отчётов, мониторов данных, инструментальных консолей, сетевых моделей, которые используются сразу после установки ArcSight ESM. В тоже время все предопределённые настройки могут изменяться и настраиваться под определённые нужды.

База данных (ArcSight DataBase), кроме обработанных сообщений информационной безопасности, хранит информацию о конфигурации системы мониторинга, пользователях системы мониторинга, группах, правах, правилах и отчётах. Сервер базы данных функцианирует на базе Oracle 10. БД Oracle устанавливается и настраивается автоматически при установке ArcSight ESM.

SmartStorage Partition Management --- модуль управления разделами (партициями) базы данны. БД разделяется на разделы оперативного хранения, разделы среднесрочного хранения, долговременного хранения, и разделы offline хранения.

Консоль управления (ArcSight Console) выполняется на APM администратора ArcSight ESM и предназначена для:
\begin{itemize}
    \item визуального контроля событий информационной безопасности;
    \item создания и настройки фильтров и правил обработки сообщений;
    \item описание правил оповещения;
    \item генерации отчётов;
    \item административных действий по созданию пользователей ArcSight ESM и прав.
\end{itemize}

Интерфейс и набор инструментальных средств консоли управления зависит от роли использующего его пользователя.

Интерфейс ArcSight Web представляет из себя web-сервер, который обеспечивает безопасный интерфейс к модулю управления ArcSight Manager с помощью web проводника. ArcSight Web предназначен для пользователей, которым нужно иметь доступ к ArcSight ESM извне защищенной сети организации. 

В системе ArcSight применяется ролевая модель доступа. Роли позволяют описывать различные уровни доступа пользователей к системе и их зоны ответственности, предоставляя различные инструментальные средства.

В ArcSight используются следующие роли:

\begin{itemize}
    \item[1]Администратор --- осуществляет установку и общую работоспособность системы;
    \item[2]Автор --- моделирует сеть, разрабатывает системные ресурсы, а также шаблоны отчётов и статьи базы данных;
    \item[3]Оператор --- отвечает за мониторинг событий и первичное расследование инцидентов;
    \item[4]Аналитик --- отвечает за детальное расследование и исправление инцидентов по запросам от операторов;
    \item[5]Менеджер безопасности --- отвечает за управление и работы в Центре безопасности;
    \item[6]Бизнес-пользователь --- использует ArcSight, чтобы устанавливать и передавать системные условия.
\end{itemize}

\subsection{Вывод}
В рамках третьей главы были описаны используемые при разработке языки программирования и инструменты, существующие компоненты программного модуля, возможности пользователей и интерфейс. Обозначены планы на будущее в рамках реализации данного программного средства. Выполнено функциональное тестирование, показавшее что все функции реализованы и работают корректно.