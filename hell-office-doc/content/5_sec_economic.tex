% !TEX root = ../hell-office.tex
\section{Технико-экономическое обоснование разработки и внедрения в эксплуатацию системы управления и мониторинга информационной безопасности компании}
\label{sec:economics}

\subsection{Характеристика разработанной системы управления и мониторинга информационной безопасности}

Система управления и мониторинга информационной безопасности компании является самостоятельной разработкой и будет использоваться для собственных нужд.

В рамках разрабатываемой системы были поставлены задачи автоматизации и/или упрощения процесса управления и мониторинга информационной безопасности компании. Благодаря системе управления и мониторинга информационной безопасности компании планируется уменьшить затраты на заработную плату специалистов по информационной безопасности компании, а так же улучшить качество обнаружения и предотвращения инцидентов информационной безопасности.

\subsection{Расчет инвестиций на разработку и внедрение в эскплуатацию системы управления и мониторинга информационной безопасности}
Расчет затрат на разработку системы управления и мониторинга информационной безопасности в данном случае будет состоять из следующих пунктов:
\begin{itemize}
    \item затраты на комплектующие изделия;
    \item затраты на основную заработную плату разработчиков;
    \item затраты на дополнительную заработную плату разработчиков;
    \item затраты на материалы для монтажа системы;
    \item затраты на монтаж технической составляющей системы;
    \item отчисления на социальные нужды;
    \item прочие затраты (амортизационные отчисления, расходы на электроэнергию, командировочные расходы, арендная плата за офисные помещения и оборудование, расходы на управление и реализацию и т.п.);
    \item плановая прибыль, включаемая в цену системы управления и мониторинга информационной безопасности.
\end{itemize}

\begin{itemize}
    \item[1] Затраты на комплектующие изделия.
Включаются затраты на приобретение в порядке производственной кооперации готовых покупных изделий и полуфабрикатов, используемых для комплектования изделий.

Расчет затрат на комплектующие изделия представлен в таблице \ref{tab1}.

\begin{table}[!h!t]
\caption{Расчет затрат на комплектующие изделия }
\label{tab1}
\centering

	\begin{tabular}{
	        | >{\raggedright}m{0.4\textwidth}
			| >{\centering\arraybackslash}m{0.14\textwidth}
			| >{\centering\arraybackslash}m{0.17\textwidth}  
			| >{\centering\arraybackslash}m{0.18\textwidth}|
			}

\hline
  Наименование комплектующиего изделия или полуфабриката & Коли-чество на единицу, шт. & Цена за единицу, руб. & Сумма, руб. \\ 

\hline
1. Сервер HP Proliant DL165R07 &	1	 & 3937,5 &	3937,5   \\ 

\hline
2. Оперативная память 2GB &	2 &	141,4 &	282,8 \\    

\hline
3. Жесткий диск HP 160GB &	2	 & 92,2 &	184,4 \\ 

\hline
4. Сервер HP Proliant DL180 &	1	& 2812,0 &	2812,0 \\ 

\hline
5. Процессор ProLiant DL160 &	1 &	1100 &	1100 \\

\hline
6. Оперативная память 2GB &	8 &	106,1 &	848,8\\

\hline
7. Корзина HP DL180G6 &	1 &	115,3 &	115,3\\


\hline
8. Жесткий диск HP 1TB &	14 &	700 &	9800\\

\hline
9. Блок питания HP 750W &	1 &	325 &	325\\

\hline
\multicolumn{3}{|l|}{Всего}    & 19405,8   \\ 
\hline
\multicolumn{3}{|l|}{Всего с учётом транспортно-заготовительных расходов }    & 21346,4\\ 
\hline

\end{tabular}
\end{table}





    \item[2]Затраты на основную заработную плату команды разработчиков.
Основная заработная плата исполнителей проекта определяется по формуле:

$$
    \text{З}_{\text{о}}=\text{К}_{\text{пр}}\cdot \sum_{i=1}^{n}{\text{З}_{\text{ч}i}\cdot t_{i}},
$$

где	$n$ -- количество исполнителей, занятых разработкой конкретного системы управления и мониторинга информационной безопасности;

    $\text{К}_{\text{пр}}$ -- коэффициент премий (1,5); 
    
    $\text{З}_{\text{ч}i}$ -- часовая заработная плата i-го исполнителя (руб.); 
    
    $t_{i}$ -- трудоемкость работ, выполняемых i-м исполнителем (ч).
    


Расчет основной заработной платы представлен в таблице \ref{tab13}.

\begin{table}[!h!t]
\caption{Расчет основной заработной платы }
\label{tab13}
\centering

	\begin{tabular}{
	        | >{\raggedright}m{0.02\textwidth}
			| >{\centering\arraybackslash}m{0.17\textwidth}
			| >{\centering\arraybackslash}m{0.17\textwidth}  
			| >{\centering\arraybackslash}m{0.12\textwidth}
			| >{\centering\arraybackslash}m{0.1\textwidth}
			| >{\centering\arraybackslash}m{0.12\textwidth}
			| >{\centering\arraybackslash}m{0.12\textwidth}|
			}

\hline
№ & Наименова-ние должности разработчика & Вид выполняемой работы & Месячная заработная плата, руб. & Часовая заработная плата, руб. & Трудо-емкость работ, ч. & Зарплата по тарифу, руб. \\ 

\hline
1 & 2 & 3 & 4 & 5 & 6 & 7   \\ 

\hline
1 & Руководитель проекта  & анализ системы   & 1300,00  & 8,125   & 100  & 812,5 \\    

\hline
2 & Специалист по информационной безопасности   & разработка системы  & 1100  & 6,875 & 200  & 1375 \\ 

\hline
3 & Системный архитектор   & разработка архитектуры взаимодействия компонентов системы  & 900  & 5,625 & 70  & 393,75 \\ 

\hline
4 & Специалист по тестированию программного обеспечения  & тестирование полномасштабной системы   & 800  & 5 & 40  & 200 \\


\hline
\multicolumn{6}{|l|}{Итого}    & 2781,25   \\ 
\hline


\hline
\multicolumn{6}{|l|}{Премия(50\%)}    & 1390,625   \\ 
\hline

\multicolumn{6}{|l|}{\begin{tabular}[c]{@{}l@{}}Итого затраты на основную заработную плату\\ разработчиков\end{tabular}}  & 4171,875  \\ 
\hline
\end{tabular}
\end{table}

\item[3]Затраты на дополнительную заработную плату команды разработчиков включает выплаты, предусмотренные законодательством о труде (оплата трудовых отпусков, льготных часов, времени выполнения государственных обязанностей и других выплат, не связанных с основной деятельностью исполнителей), и определяется по формуле:

$$
    \text{З}_{\text{д}} = \frac{\text{З}_{o}\cdot H_\text{д}}{100},
$$


где $H_\text{д}$ -- норматив дополнительной              заработной платы(20 \%);

   $\text{З}_{\text{o}}$ -- затраты на основную заработную плату (руб.).

\newpage

Дополнительная заработная плата составит:

$$
\text{З}_{\text{д}} = \frac{4171,875 \cdot 20}{100} = 834,375 \text{ руб}.
$$



\item[4]Расчет затрат на материалы для монтажа системы осуществляются в табличной форме и представлены в таблице \ref{tab24}


\begin{table}[!h!t]
\caption{Расчет затрат на материалы для монтажа }
\label{tab24}
\centering

	\begin{tabular}{
	        | >{\raggedright}m{0.30\textwidth}
			| >{\centering\arraybackslash}m{0.15\textwidth}
			| >{\centering\arraybackslash}m{0.15\textwidth}  
			| >{\centering\arraybackslash}m{0.15\textwidth}
			| >{\centering\arraybackslash}m{0.12\textwidth}|
			}

\hline
 Наименование материала & Единица измерения & Норма расхода & Цена за единицу & Сумма, руб. \\ 

\hline
1. Кабель сетевой UTP-CAT5E & м & 30 & 3 & 90   \\ 

\hline
2. Короб декоративный белый & м & 10 & 2,5  & 25 \\    

\hline
3.Коннекторы RJ-45 & шт  & 55  & 1,5  & 82,5 \\ 

\hline
4.Шкаф серверный напольный 42U ЦМО ШТК-С-2  & шт  & 1  & 1900  & 1900 \\ 

\hline
5.Розетка электрическая белая & шт  & 5   & 3 & 15 \\

\hline 
6.Розетка компьютерная белая внешняя & шт & 2 & 8 & 16\\

\hline
7.1. Кабель КМВЭВ1х2х0,75 & м & 10 & 3,2 & 32 \\
\hline
\multicolumn{4}{|l|}{Всего}    & 2160,5   \\ 
\hline

\hline
\multicolumn{4}{|l|}{Всего с учетом транспортных расходов }    & 2592,6   \\ 
\hline

\end{tabular}
\end{table}



\item[5]Затраты на монтаж технической составляющей системы.
Расчет заработной платы на монтаж системы осуществляется в табличной форме и представлен в таблице \ref{tab15}.

\newpage

\begin{table}[!h!t]
\caption{Расчет заработной платы на монтаж системы}
\label{tab15}
\centering

	\begin{tabular}{
	        | >{\raggedright}m{0.22\textwidth}
			| >{\centering\arraybackslash}m{0.13\textwidth}
			| >{\centering\arraybackslash}m{0.13\textwidth}  
			| >{\centering\arraybackslash}m{0.13\textwidth}
			| >{\centering\arraybackslash}m{0.10\textwidth}
			| >{\centering\arraybackslash}m{0.13\textwidth}|
			}

\hline
Наименование категории работника и должности & Чис-ленность исполнителей, чел. & Месячная заработная плата, руб. & Дневная заработная плата, руб. & Трудо-емкость работ, дн. & Сумма, руб. \\ 

\hline
1.Главный инженер & 1 & 1000 & 47,6 & 2 & 95,2   \\ 

\hline
2.Электро-монтер & 1 & 800   & 38  & 3   & 114,2 \\    

\hline
\multicolumn{5}{|l|}{Итого}    & 204,5   \\ 
\hline


\hline
\multicolumn{5}{|l|}{Премия(40\%)}    & 83,8   \\ 
\hline

\hline
\multicolumn{5}{|l|}{Всего основная заработная плата }    & 288,3   \\ 
\hline
\end{tabular}
\end{table}

\newpage

Дополнительная заработная плата составит:

$$
\text{З}_{\text{д}} = \frac{288,3 \cdot 20}{100} = 57,7 \text{ руб}.
$$


\item[6]Отчисления в фонд социальной защиты и обязательного страхования (в фонд социальной защиты населения и на обязательное страхование) определяются в соответствии с действующими законодательными актами определяются по формуле: 

$$
    \text{Р}_{\text{соц}} = \frac{(\text{З}_{o} + \text{З}_{\text{д}})\cdot H_\text{соц}}{100},
$$
 
где $H_\text{соц}$ -- норматив отчислений в фонд социальной защиты населения и на обязательное страхование (34,6 \%).
 
$$
 \text{Р}_{\text{соц}} = \frac{(4171,875 + 834,375 + 288,3 + 57,7)\cdot 34,6}{100} = 1851,88 \text{ руб}.
$$

\item[7]Прочие затраты включаются в себестоимость разработки системы управления и мониторинга информационной безопасности в процентах от затрат на основную заработную плату команды разработчиков определяются по формуле:

$$
    \text{Р}_{\text{пр}} = \frac{\text{З}_{o} \cdot \text{Н}_{\text{пр}}}{100},
$$

где $\text{Н}_{\text{пр}}$ -- норматив прочих затрат (70 \%).

$$
 \text{Р}_{\text{пр}} = \frac{4460,2 \cdot 70}{100} =  3122,12 \text{ руб}.
$$

\item[8]Общая сумма затрат на разработку системы управления и мониторинга информационной безопасности находится путем суммирования всех рассчитанных статей затрат, и определяется по формуле:
$$
    \text{З}_{\text{р}} = \text{3}_{\text{к}} +  \text{З}_{o} + \text{З}_{\text{д}} + \text{Р}_{\text{соц}} + \text{Р}_{\text{пр}},
$$

$$
 \text{З}_{\text{р}} =21346,4 + 4171,875 +346 + 834,375 + 1732,1625 + 2920,312 =  31351,12 \text{ руб}.
$$
\item[9]Отпускная цена системы управления и мониторинга информационной безопасности определяется по формуле:

$$
    \text{Ц}_{\text{пс}} = \text{З}_{p},
$$

$$
\text{Ц}_{\text{пс}} =  31351,12 \text{ руб}.
$$


Формирование цены системы управления и мониторинга информационной безопасности на основе затрат представлено в таблице \ref{tab2}.

\begin{table}[!h!t]
\centering
\caption{Формирование цены на основе затрат на разработку системы управления и мониторинга информационной безопасности}
\label{tab2}
\begin{tabular}{|l|c|}
\hline
\begin{tabular}[c]{m{0.76\textwidth}}{Статья затрат}\end{tabular} & Сумма, руб. \\ \hline
\begin{tabular}[c]{m{0.76\textwidth}}{Затраты на комплектующие изделия}\end{tabular} & 21346,4\\ \hline
\begin{tabular}[c]{m{0.76\textwidth}}Основная заработная плата команды разработчиков\end{tabular} & 4171,875      \\ \hline
\begin{tabular}[c]{m{0.76\textwidth}}Дополнительная заработная плата команды\\ разработчиков\end{tabular} & 834,375      \\ \hline
\begin{tabular}[c]{m{0.76\textwidth}}Затраты на материал для монтажа\end{tabular} & 2592,6      \\ \hline
\begin{tabular}[c]{m{0.76\textwidth}}Основная заработная плата на монтаж системы\end{tabular} & 288,3      \\ \hline
\begin{tabular}[c]{m{0.76\textwidth}}Дополнительная заработная платa на монтаж системы\end{tabular} & 57,7      \\ \hline
\begin{tabular}[c]{m{0.76\textwidth}}Отчисления в фонд социальной защиты и\\ обязательного страхования\end{tabular} & 1851,88     \\ \hline
\begin{tabular}[c]{m{0.76\textwidth}}Прочие затраты\end{tabular} & 3122,12      \\ \hline
\begin{tabular}[c]{m{0.76\textwidth}}Итого\end{tabular} & 34265,25     \\ \hline
\end{tabular}
\end{table}
\end{itemize}


\newpage

\subsection{Расчет экономического эффекта от использования системы управления и мониторинга информационной безопасности}

Использование системы управления и мониторинга информационной безопасности в первую очередь централизует и нормализует поступление информации со всевозможных источников-обьектов системы безопасности, тем самым облегчая и улучшая работу отдела информационной безопасности. Быстрый анализ инцидентов и реагирование на них понижает вероятность утечек данных, проникновений злоумышленников во внутреннюю сеть организации и повышает производительность отдела безопасности путем снижения трудоемкости ряда процессов. Использование системы скажется на высвобождении рабочего времени персонала, который,  впоследстви, станет обслуживать систему. Экономический эффект при использовании системы будет рассчитываться следующим образом.

Экономия затрат на заработную плату при использовании системы управления и мониторинга информационной безопасности для организации-заказчика определяется по формуле:

$$
    \text{Э}_{\text{зп}} = \text{К}_{\text{пр}}\cdot (t_{\text{p}}^{\text{без пс}}-t_{\text{p}}^{\text{с пс}})\cdot \text{Т}_{\text{ч}}\cdot N_{\text{п}}\cdot  (1+\frac{\text{Н}_{\text{д}}}{100})\cdot (1+\frac{\text{Н}_{\text{но}}}{100}), 
$$

где $N_{\text{п}}$ -- плановый объем работ; 

    $t_{\text{p}}^{\text{без пс}}, t_{\text{p}}^{\text{с пс}}$ -- трудоемкость выполнения работы до и после внедрения системы управления и мониторинга информационной безопасности (ч.); 
    
    $\text{Т}_{\text{ч}}$ -- часовая тарифная ставка, соответствующая разряду выполняемых работ (3,75 руб./ч.); 
    
    $\text{К}_{\text{пр}}$ -- коэффициент премий (1,5); 
    
    $\text{Н}_{\text{д}}$ -- норматив дополнительной заработной платы (20\%); 
    
    $\text{Н}_{\text{но}}$ -- ставка отчислений от заработной платы, включаемых в себестоимость (34,6\%).
 
$$
    \text{Э}_{\text{З}} = 1,5 \cdot (7-4)\cdot 3,75\cdot 750 \cdot  (1+\frac{20}{100})\cdot (1+\frac{34,6}{100}) =  20442,375 \text{ руб}.
$$
    
Экономический эффект при использовании системы управления и мониторинга информационной безопасности будет рассчитываться по формуле:

$$
    \Delta \text{П}_{\text{ч}} = (\text{Э}_{\text{з}} - \Delta \text{З}_{\text{тек}}) \cdot (1 - \text{Н}_{\text{п}}), 
$$

где	$\text{Э}_{\text{з}}$ -- экономия текущих затрат, полученная в результате применения системы управления и мониторинга информационной безопасности (руб.); 

    $\Delta \text{З}_{\text{тек}}$ -- прирост текущих затрат, связанных с использованием системы управления и мониторинга информационной безопасности ( 15\% отпускной стоимости системы ); 
    
    $\text{Н}_{\text{п}}$ --  ставка налога на прибыль, в соответствии с действующим законодательством (18\%).

$$
    \Delta \text{П}_{\text{ч}} = (20442,375 - 6511,08) \cdot (1 - 0,18) = 11423,67 \text{ руб}.
$$


\subsection{Расчёт показателей экономической эффективности разработки и использования системы управления и мониторинга информационной безопасности}

Так как сумма инвестиций больше суммы годового экономического эффекта, то экономическая целесообразность инвестиций в разработку и использование программного продукта осуществляется на основе расчёта и оценки следующих показателей:
\begin{enumerate}
    \item[-]	чистый  дисконтированный доход (ЧДД );
    \item[-] срок окупаемости инвестиций ($\text{Т}_{\text{ок}} $);
    \item[-] рентабельность инвестиций ($\text{Т}_{\text{и}} $).
\end{enumerate}
Так как приходится сравнивать разновременные результаты (экономический эффект) и затраты (инвестиции в разработку программного продукта), необходимо привести их к единому моменту времени – началу расчётного периода, что обеспечивает их сопоставимость.
Для этого необходимо использовать дисконтирование путём умножения соответствующих результатов и затрат на коэффициент дисконтирования соответствующего года t, который определяется по формуле



$$
     \alpha_\text{t} = \frac {1}{(1 + {\text{Е}_\text{н}})^{\text{t}}}
$$

где  $\text{Е}_\text{н}$  — норма дисконта (в долях единиц), равная или больше средней процентной ставки по банковским депозитам, действующей на момент осуществления расчётов, 0,12;

t – порядковый номер года периода реализации инвестиционного проекта (предполагаемый период использования разрабатываемого ПО пользователем и время на разработку).

$$
    \alpha_\text{t} = \frac {1}{(1 + 0,12)} = 0,8928
$$


Чистый дисконтированный доход рассчитывается по формуле


$$
 \text{ЧДД} = \sum_{t=1}^{n}(\text{P}_\text{t} - \text{3}_\text{t}) \cdot\alpha_\text{t}
$$

где n – расчётный период, лет;

 $ \text{P}_\text{t} $ – результат (экономический эффект – прибыль или чистая прибыль), по-лученный в году t, р.;
 
$\text{3}_\text{t}$ – затраты (инвестиции – затраты на разработку (модернизацию) или на приобретение и внедрение ПО) в году t, р.

Расчёт показателей эффективности инвестиций представлен в таблице \ref{tab34} 


\begin{table}[!h!t]
\caption{Расчёт чистого дисконтированного дохода и срока окупаемости инвестиций в разработку и внедрения программно-аппаратного комплекса, р.}
\label{tab34}
\centering
	\begin{tabular}{
	        | >{\raggedright}m{0.23\textwidth}
			| >{\centering\arraybackslash}m{0.10\textwidth}
			| >{\centering\arraybackslash}m{0.14\textwidth}  
			| >{\centering\arraybackslash}m{0.13\textwidth}
			| >{\centering\arraybackslash}m{0.12\textwidth}
			| >{\centering\arraybackslash}m{0.12\textwidth}
			|
			}

\hline
Наименование показателя & Усл. обоз. & \multicolumn{4}{|l|}{Расчетный период, год}  \\ 
\hline
Результат & & 1-ый & 2-ой & 3-ий & 4-ый\\

\hline
1. Прирост результата(чистой прибыли) & $ \text{P}_\text{t} $ & 11423,67  & 11423,67 & 11423,67  & 11423,67\\ 

\hline
2.Коэффициент дисконтирования & $\alpha_\text{t}$ & 1 & 0,8928  & 0,7971  &0,7118 \\    

\hline
3.Результат с учетом фактора времени & $\text{P}_\text{t}\alpha_\text{t}$ & 11423,67 & 10199,05 & 9104,66 &8131,37 \\

\hline
Затраты (инвестиции) & & & & & \\
\hline
4.Инвестиции в разработку ПО & 3 & 31351,12 & - & - &- \\

\hline
5.Инвестиции с учетом фактора времени &$\text{3}_\text{t}\alpha_\text{t}$ & 31351,12 & - & -&- \\

\hline
6.Чистый дисконтированный доход по годам & $\text{ЧДД}_\text{t}$ & -19927,45 & 10199,05 &9104,66 & 8131,37 \\
\hline
7.ЧДД нарастающим итогом & ЧДД & -19927,45 &-9728,4 & -623,74 & +7507,63\\
\hline

\end{tabular}
\end{table}

\newpage

Рентабельность инвестиций определяется по формуле 

$$ 
\text{P}_\text{и} = \frac{\text{П}_\text{чср}}{3} \cdot 100\%
$$

где $\text{П}_\text{чср}$ – среднегодовая величина чистой прибыли за расчётный период, р., которая определяется по формуле

$$
\text{П}_\text{чср} = \frac{\sum_{t=1}^{n}\text{П}_\text{чt}}{n}
$$

$$
 \text{П}_\text{чср} = \frac{45694,68}{4} = 11423,67 \text{ руб}.
$$

где $\text{П}_\text{чt}$ – чистая прибыль, полученная в году $t$, р.

$$
 \text{P}_\text{и} = \frac{11423,67}{43891,6} \cdot 100\% = 26\%
$$

\subsection{Вывод}
В результате технико-экономического обоснования разработки и внедрения в эксплуатацию системы управления и мониторинга информационной безопасности были получены следующие значения показателей эффективности:

\begin{enumerate}
\item[1] прирост среднегодовой чистой прибыли организации-заказчика составит 11423,67 руб.;
\item[2] затраты на разработку системы управления и мониторинга информационной безопасности для организации-заказчика окупятся на четвертый год  его использования без учета фактора времени и на пятый с учетом фактора времени;
\item[3] инвестиции на разработку системы управления и мониторинга информационной безопасности будут экономически эффективными, т.к. рентабельность инвестиций составляет 26 \%.
\end{enumerate}

Таким образом, разработка и применение системы управления и мониторинга информационной безопасности является эффективной и данные инвестиции осуществлять целесообразно.