\sectioncentered*{ВВЕДЕНИЕ}
\addcontentsline{toc}{section}{ВВЕДЕНИЕ}

В современном автомобилестроении особое внимание уделяется повышению безопасности и комфорта вождения. Одним из ключевых аспектов, влияющих на безопасность, является эффективное освещение дорожного полотна, особенно в условиях плохой видимости и при выполнении маневров, таких как повороты. Традиционные системы освещения, использующие статичное направление света фар, не всегда обеспечивают достаточную видимость на поворотах, что может привести к аварийным ситуациям. В связи с этим актуальной задачей становится разработка и внедрение систем, способных динамически изменять направление света фар в зависимости от угла поворота руля и скорости движения автомобиля.

В последние годы на рынке автомобильных технологий наблюдается активное развитие интеллектуальных систем освещения, таких как адаптивные фары (Adaptive Front-lighting System, AFS). Эти системы позволяют автоматически регулировать направление светового потока в зависимости от дорожной ситуации, что значительно улучшает видимость и снижает риск возникновения аварий. Однако, несмотря на очевидные преимущества, такие системы требуют точного управления и интеграции с другими электронными системами автомобиля, что делает их разработку сложной технической задачей.

Целью данного дипломного проекта является разработка блока управления направлением света фар автомобиля при поворотах. Данный блок должен обеспечивать автоматическую корректировку угла наклона фар в зависимости от угла поворота руля и скорости движения, что позволит улучшить освещение дорожного полотна в поворотах и повысить безопасность вождения.

Достижение поставленной цели реализуется посредством выполнения следующих задач:
\begin{itemize}
    \item проведение анализа существующих решений в области адаптивных систем освещения и выявление их преимуществ и недостатков;
    \item разработка алгоритма управления направлением света фар на основе данных о угле поворота руля и скорости автомобиля;
    \item проектирование аппаратной части блока управления, включая датчики, исполнительные механизмы и микроконтроллер;
    \item разработка программного обеспечения для управления системой и проведение тестирования разработанного блока в условиях, приближенных к реальным.
\end{itemize}

Разрабатываемый блок управления направлением света фар может быть интегрирован в современные автомобили, оснащенные электронными системами управления. Это позволит повысить безопасность вождения, особенно в условиях недостаточной видимости, и улучшить комфорт для водителя. В качестве инструментов для разработки будут использованы современные средства проектирования электронных систем, такие как CAD-программы для проектирования печатных плат, среды разработки программного обеспечения для микроконтроллеров, а также специализированное оборудование для тестирования и отладки.

Таким образом, данный проект направлен на создание инновационного решения, которое может быть внедрено в автомобильную промышленность для повышения безопасности и комфорта вождения.
